%%%%%%%%%%%%%%%%%%%%%%%%%%%%%%%%%%%%%%%%%%%%%%%%%%

\chapter{Ejercicios propuestos}

%%%%%%%%%%%%%%%%%%%%%%%%%%%%%%%%%%%%%%%%%%%%%%%%%%

\section{Matrices}

\begin{enumerate}
\item Dado el vector $x=[3,1,5,7,9,2,6]$, intentar prever a mano el
  resultado de las siguientes operaciones elementales\@.

  \begin{enumerate}
  \item \texttt{x(3)}
  \item \texttt{x(1:7)}
  \item \texttt{x(1:end)}
  \item \texttt{x(1:end-1)}
  \item \texttt{x(6:-2:1)}
  \item \texttt{x({[}1,6,2,1,1{]})}
  \item \texttt{sum(x)}
  \end{enumerate}
\item Dada la matriz $A=[2,4,1;6,2,7;3,5,9],$ calcular:

  \begin{enumerate}
  \item El vector \texttt{x1} que es la primera fila de $A$.
  \item Asignar las 2 primeras filas de $A$ a un vector llamado
    \texttt{y}.
  \item Calcular la suma de los elementos de las columnas de $A$.
  \item Calcular la suma de los elementos de las filas de $A$.
  \item Calcular la desviación estándar de cada columna de $A$.
  \end{enumerate}
\item Dada la matriz:
$$A=\left(\begin{array}{cccc}
      2 & 7 & 9 & 7\\
      3 & 1 & 5 & 6\\
      8 & 1 & 2 & 5\\
      4 & 2 & 8 & 7\end{array}\right)$$


  \begin{enumerate}
  \item Asignar los elementos impares de la matriz $A$ a otra matriz
  \item Asignar los elementos pares de la matriz $A$ a otra matriz
  \item Calular la raíz cuadrada de cada elemento de la matriz $A$
  \end{enumerate}
\end{enumerate}

\section{Programación}

\begin{enumerate}
\item Dado el vector \texttt{x={[}3 15 9 2 -1 0 -12 9 6 1{]}} escribir
  los comandos que darán los siguientes resultados, preferiblemente en
  un único comando:

  \begin{enumerate}
  \item Devolver un vector que convierta los elementos de \texttt{x}
    en 0
  \item Convertir los elementos múltiples de 3 del vector en 3.
  \item Multiplicar los valores impares de x por cinco.
  \item Asignar los valores de módulo mayor a 10 a otro vector.
  \item Convertir los elementos que son menores a la media en cero
  \item Convertir los elementos que son mayores a la media en la
    media.
  \end{enumerate}
\item Escribir una función cuyo resultado sea:
$$ y(x)\left\{
    \begin{array}{ccc}
      2 & \qquad & x<6\\
      x-4 & \qquad & 6\leq x\leq20\\
      36-x & \qquad & 20\leq x\leq36\end{array}\right.$$
 representarla
  gráficamente para ver que tiene forma triangular con \texttt{fplot}.
\item Calcular el número $\pi$ mediante la seguiente serie:
$$\frac{\pi^{2}-8}{16}=\sum_{n=1}^{\infty}\frac{1}{(2n-1)^{2}(2n+1)^{2}}$$
  ¿Cuántos términos son necesarios para llegar a una precisión de
  $1\times10^{12}$?  ¿Cuánta es la precisión de la suma de 100
  términos?
\item Podemos calcular una serie de Fibonacci mediante la siguiente
  relación recursiva:
$$ F_{n}=F_{n-1}+F_{n-2}$$
con $F_{0}=F_{1}=1$.
  Calcular los 10 primeros números de la serie de Fibonacci. La
  relación $\frac{F_{n}}{F_{n-1}}$ se va aproximando al número áureo
  $\varphi=\frac{1+\sqrt{5}}{2}$ a medida que $n$ aumenta. ¿En qué
  término de la serie de Fibonacci nos acercamos al número áureo con
  una precisión de $1\times10^{12}$?
\item Los polinomios de Legendre $P_{n}(x)$ se definen con la
  siguiente relación recurrente:
$$  (n+1)P_{n+1}(x)-(2n+1)P_{n}(x)+nP_{n-1}(x)=0$$ con $P_{0}(x)=1$,
  $P_{1}(x)=x$ y $P_{2}(x)=\frac{3x^{2}-1}{2}$.  Cacular los tres
  siguientes polinomios y representarlos gráficamente en el intervalo
  {[}-1,1{]}
\item Se plantea el siguiente algoritmo para calcular el número $\pi$.
  Primero iniciar las constantes: $a=1$, $b=\frac{1}{\sqrt{2}}$,
  $t=\frac{1}{4}$ y $x=1$. Repetir la siguiente secuencia de comandos
  hasta que llegamos
  a la precisión deseada:\\
  \texttt{y = a}~\\
  \texttt{a = (a+b)/2}~\\
  \texttt{b = sqrt(b{*}y)}~\\
  \texttt{t = t-x{*}(y-a)\textasciicircum{}2}~\\
  \texttt{x = 2{*}x}\\
  Por último se toman los valores de $a$, $b$, y $t$ para estimar
  $\pi$ con:\\
  \texttt{pi\_est = ((a+b)\textasciicircum{}2)/(4{*}t)}~\\
  ¿Cuántas iteraciones son necesarias para llegar a una precisión de
  $1\times10^{12}$?
\item Escribir un programa que pida por línea de comandos un entero
  ($n$)
  y luego calcule lo siguiente.\\
  1. Mientras el valor de $n$ sea mayor que 1, cambiar el entero por
  la mitad de su valor si el entero es par; si es impar cambiarlo por
  tres veces su valor mas 1.\\
  2. Repetir estas operaciones cíclicamente y obtener el número de
  iteraciones hasta que el algoritmo se para en 1. Por ejemplo, para
  $n=10$ la secuencia es 5, 16, 8, 4, 2 y 1; entonces el número de
  iteraciones
  han sido 6.\\
  3. Representar la longitud de las secuencias en función del entero
  de entrada de 2 a 30 e intentar descubrir un patrón. Si no se deduce
  intentarlo con más números.\\
  4. ¿Hay algún número para el que la secuencia sea interminable?
\item Escribir un programa que convierta números romanos en decimales.
  La
  tabla de conversion es: \\
  \begin{minipage}[c]{1\linewidth}%
\centering{}\begin{tabular}{|c|c|}
\hline 
Romanos&
Decimales\tabularnewline
\hline
\hline 
I&
1\tabularnewline
\hline 
V&
5\tabularnewline
\hline 
X&
10\tabularnewline
\hline 
L&
50\tabularnewline
\hline 
C&
100\tabularnewline
\hline 
D&
500\tabularnewline
\hline 
M&
1000\tabularnewline
\hline
\end{tabular}\end{minipage}%

\item Escribir un programa que haga la función inversa, que pase de números
decimales a romanos. En ambos casos el número se pedirá por pantalla.
\end{enumerate}

\section{Álgebra lineal}

\begin{enumerate}
\item Resolver el siguiente sistema de ecuaciones lineales:$$
  \begin{array}{ccccccc}
    x & - & 2y & + & 3z & = & 9\\
    -x & + & 3y &  &  & = & 4\\
    2x & - & 5y & + & 5z & = & 17\end{array}$$

\item Sean $B=\{(1,0,0),(0,1,0),(0,0,1)\}$ y
  $B^{\prime}=\{(1,1,0),(1,-1,0),(0,0,1)\}$ bases en $\mathbb{R}^{3}$,
  y sea $$ A=\left[\begin{array}{ccc}
      1 & 3 & 0\\
      3 & 1 & 0\\
      0 & 0 & -2\end{array}\right]$$ la matriz de la aplicación lineal
  $T:\mathbb{R}^{3}\rightarrow\mathbb{R}^{3}$ en la base $B$, base
  cartesiana ortogonal se pide:

  \begin{enumerate}
  \item La matriz de cambio de base $P$ de $B^{\prime}$ a $B$
  \item La matriz de camnio de base $P^{-1}$de $B$ a $B^{\prime}$
  \item La matriz $A^{\prime}$ que es la matriz $A$ expresada en la
    base $B^{\prime}$.
  \item Sea$$ [v]_{B^{\prime}}=\left[\begin{array}{c}
        1\\
        2\\
        3\end{array}\right]$$
 encontrar $[v]_{B}$ y $[T(v)]_{B}$
  \item Encontrar $[T(v)]_{B^{\prime}}$ de dos maneras, primero como
    $P^{-1}[T(v)]_{B}$ y luego como $A^{\prime}[v]_{B^{\prime}}$.
  \end{enumerate}
\item Crear la matriz de $5\times5$ $T$ tal que:
$$ T_{kl}=k-l$$ y
  calcular sus autovalores y autovectores
\item Un proceso para encriptar un mensaje secreto es usar cierta
  matriz cuadrada con elementos enteros y cuya inversa es también
  entera. Se asigna un número a cada letra (A=1, B=2... espacio=28) y
  se procede como sigue a continuación. Supongamos que la matriz es:
  $$A=\left(\begin{array}{ccccc}
      1 & 2 & -3 & 4 & 5\\
      -2 & -5 & 8 & -8 & 9\\
      1 & 2 & -2 & 7 & 9\\
      1 & 1 & 0 & 6 & 12\\
      2 & 4 & -6 & 8 & 11\end{array}\right)$$ y el mensaje que
  queremos enviar es {}``ALGEBRA LINEAL''. Para codificar el mensaje
  romperemos la cadena de números en vectores de cinco elementos y los
  codificaremos.

  \begin{enumerate}
  \item ¿Cuál será la cadena de datos encriptada que enviaremos?
  \item Descifrar el mensaje: 48 64 -40 300 472 61 96 -90 346 538 16
    -5 71 182 332 68 131 -176 322 411 66 125 -170 301 417.
  \end{enumerate}
\end{enumerate}

\section{Cálculo y Ecuaciones Diferenciales Ordinarias.}

\begin{enumerate}
\item Hallar el área $A$ de la región del plano comprendida entre la
  curva
$$ y=\frac{x^{2}-1}{x^{2}+1}$$
 y su asíntota. ($2\pi$)
\item Hallar la longitud $s$ del arco de cicloide
$$ \left\{
    \begin{array}{ccc}
      x & = & a(t-\sin t)\\
      y & = & a(t-\cos t)\end{array}\right.\quad0\leq t\leq2\pi$$
  ($8a$)
\item Hallar el centro de masas de la cardioide
$$
  \rho=a(1+\cos\theta)$$ (Sobre el eje de simetría y en
  $x=\frac{4a}{5}$)
\item El péndulo esférico tiene de ecuaciones:$$
  \ddot{\theta}=\frac{l\dot{\varphi}\sin\theta\cos\theta-g\sin\theta}{l}$$
  $$
  \ddot{\varphi}=\frac{-2\dot{\varphi}\dot{\theta}\cos\theta}{\sin\theta}$$
  Integrarlas con una condición inicial cualquiera y representar
  gráficamente las trayectorias.
\item Resolver el sistema de ecuaciones diferenciales siguiente:$$
  \begin{array}{ccc}
    \frac{dx}{dt} & = & ax+bxy-cx^{2}\\
    \frac{dy}{dt} & = & dxy-ey\end{array}$$
  donde $a=0.05$, $b=0.0002$, $c=0.00001$, $d=0.0003$ y $e=0.06$.
  Crear el mapa de fases linealizando la ecuación y utilizando el comando
  \texttt{quiver}. Resolverlo para $t\in[0,300]$ con distintos valores
  iniciales y ver que las soluciones siguen el camino marcado por el
  gradiente numérico.
\item Resolver el siguiente sistema que representa un modelo
  depredador-presa:$$
  \begin{array}{ccc}
    x_{1}^{\prime} & = & x_{1}(15-x_{2})\\
    x_{2}^{\prime} & = & x_{2}(-15+x_{1}-x_{2})\end{array}$$
  con condición inicial $x_{1}(0)=10$ y $x_{2}(0)=15$. Si los depredadores
  vienen dados por $x_{1}$, las presas por $x_{2}$ y el tiempo está
  medido en meses, ¿cuántos predadores y presas hay al cabo de un año?
  ¿Se extingue alguna de las especies en algún momento? Representar
  $x_{1}-x_{2}$ y $x_{1}$ en función de $x_{2}$.
\end{enumerate}

\section{Estadística y análisis de datos}

\begin{enumerate}
\item Generar una muestra de 1000 elementos que sigan una distribución
  normal de media $10$ y desviación típica $9$.
\item Generar una muestra de 1000 elementos que sigan una distribución
  uniforme de media cero y desviación típica $10$.
\item Calcular y representar los caminos de un grupo de partículas con
  movimiento aleatorio confinadas por un par de barreras $B^{+}$ y
  $B^{-}$ unidades del orígen que es desde donde salen las partículas.
  Un movimiento aleatorio se calcula mediante la fórmula;
$$ x_{j+1}=x_{j}+s$$ donde $s$ es el número obtenido de una
  distribución normal estandarizada de números aleatorios según la
  función \texttt{randn}. Por ejemplo, N movimientos de la partícula
  se calcularían con el fragmento de código
  siguiente:\\
  \texttt{x(1)=0;}~\\
  \texttt{for j = 1:N}~\\
  \texttt{x(j+1) = x(j) + randn(1,1);}~\\
  \texttt{end}~\\
  Se cambiarán las condiciones de contorno de la siguiente manera:\\
  1. Reflexión. Cuando una partícula se encuentre fuera de la frontera
  se devolverá al interior del dominio como si hubiera rebotado en una
  pared\\
  2. Absorción. La partícula muere cuando entra en contacto con la pared.\\
  3. Absorción parcial. Es la combinación de los dos casos previos.
  La partícula rebota en la pared y la perfección de la colisión
  depende
  de una determinada distribución de probabilidad.\\
  Calcular una serie relevante de trayectorias y calcular:

  \begin{enumerate}
  \item La posición media de las partículas en función del tiempo.
  \item La desviación típica de las posiciones de las partículas en
    función del tiempo.
  \item ¿Influyen las condiciones de contorno en las distribuciones?
  \item Para los casos de absorción y absorción parcial representar
    gráficamente el número de partículas supervivientes en función del
    número de saltos temporales.
  \end{enumerate}
\item Desarrollar en serie de Fourier la función
  $f(x)=x^{2}\qquad-\pi<x<\pi$.
\end{enumerate}

\section{Control automático}

\begin{enumerate}
\item Determinar la respuesta a un impulso y a un escalón del sistema expresado
mediante la función discreta siguiente:$$
y(n)=x(n)-2\cos(\pi/8)y(n-1)+y(n-2)$$

\end{enumerate}
