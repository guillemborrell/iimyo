\chapter{Extender Octave con otros
  lenguajes.\label{sec:Extender-Octave-con}}

\emph{Gran parte de esta sección, sobre todo los ejemplos, se basa en
  el tutorial {}``Dal segno al Coda'' de Paul Thomas. Podeis
  encontrarlo en el Wiki del proyecto Octave.}

Octave está escrito en C++, además es el lenguaje desde el que es más
sencillo extenderlo. Ahora deberíamos preguntarnos cuándo queremos
extender Octave. Si Matlab es un lenguaje de programación completo
deberíamos ser capaces de programar cualquier cosa. El problema es que
programar cualquier cosa no se hace a cualquier precio, los bucles son
un ejemplo de ello. Imaginemos que tenemos un programa que va
demasiado lento, que casi todo el tiempo de proceso se debe a la
evaluación de una función miles de millones de veces y el coste de
implementarla en otro lenguaje no es demasiado grande. La solución a
nuestro problema será escribir la función en C++ o en Fortran y luego
hacer un interfaz para que Octave pueda entenderla.

Para hacerlo antes tendremos que asegurarnos que disponemos de las
cabeceras (headers) de Octave. No se encuentran en algunos paquetes y
puede ser que tengamos que instalarlos a parte. Preparar la
infraestructura es un poco complejo, también necesitaremos un
compilador en el entorno accesible por consola, preferentemente los de
la colección gcc (gcc, g++, g77 y gfortran).

Las cabeceras sirven tanto para escribir pequeñas funciones como para
crear interfaces completos a librerías escritas en C, C++ o Fortran.
\texttt{mkoctfile} es la herramienta básica del desarrollo de Octave,
todos los archivos terminados en \texttt{.oct} han sido creados con
dicha función. Algunas de las librerías utilizadas son FFTW para las
transformadas rápidas de Fourier, ODEPACK para resolver ecuaciones
diferenciales ordinarias, UMFPACK para las matrices sparse y ATLAS
para el cálculo matricial. Los respectivos interfaces son una gran
fuente de información de cómo adecuar Octave a nuestras necesidades.


\section{Una advertencia antes de empezar a programar.}

Es posible que cuando probemos alguno de los ejemplos que se expondrán
a continuación recibamos un \emph{segmentation fault} o un resultado
sin ninguna lógica en vez de lo esperado. Muchas veces los programas
no son compilar y listo, los programadores experimentados en C y C++
saben de lo que hablo. Otra de las ventajas de los lenguajes
interpretados es que el intérprete es un binario \emph{standalone}%
\footnote{Este término informático proveniente de la expresión
  \emph{to stand alone} es el adjetivo que recibe un programa que
  despues de ser compilado ya no tiene ninguna dependencia externa;
  puede tenerlas en tiempo de compilación pero no en tiempo de
  ejecución. En un sistema operativo tipo UNIX esto es imposible
  porque todo depende como mínimo de la librería estándar de C.%
}. Octave es un programa que depende de varias librerías
externas, glibc, atlas, readline... Todas estas librerías deben ser
compiladas previamente de tal modo que sean compatibles con el
intérprete de Octave. Es normal que alguna de las dependenicas se
construya con un compilador distinto al que se utilice para Octave.
Normalmente no tendremos ningún problema pero esto no implica que no
se produzcan jamás.

Un ejemplo de ellos es cuando intentamos añadir código externo a
Octave con el programa mkoctfile. Lo que estamos haciendo es compilar
un archivo escrito en C++ con el compilador del sistema que puede ser
distinto al que se utilizó para Octave. Es incluso probable que los
compiladores sean incompatibles como por ejemplo el GCC4 y el intel
C++ compiler. Debemos tener en cuenta que Octave es a partir de ahora
un verdadero entorno de desarrollo, único y monolítico. Si queremos
ponernos a jugar de verdad lo mejor será que compilemos nosotros
mismos Octave con el compilador del sistema. Si somos incapaces de
hacerlo instalaremos el compilador con el que se construyó nuestro
Octave así como las librerías propias de C y C++ sobre las que
funciona.

Si podemos tener problemas con los compiladores de C++ con los de
Fortran debemos prepararnos para sufrir de lo lindo. Octave se compila
con gcc3, sin soporte para Fortran 95. Si queremos utilizar código en
este dialecto tendremos que utilizar gcc4 pero entonces las versiones
de las librerías de C++ serán incompatibles. Disponemos de dos
opciones:

\begin{enumerate}
\item Seguir con gcc3, escribir código en Fortran 77 y compilar con
  g77.
\item Pasarnos a la rama inestable de Octave, compilarla con gcc4 y
  gfortran y rezar para que no rompa nada.
\end{enumerate}
Por suerte estos problemas desaparecerán cuando gcc4 se consolide.


\section{Extender Octave con C++}

Una vez tengamos toda la infraestructura preparada podemos acoplar
cualquier función escrita en C++ a Octave. La diferencia entre los
archivos \texttt{.oct} y los \texttt{.mex} de Matlab es que en este
caso contamos con cabeceras que nos permiten manipular directamente
los argumentos de Octave. El header principal es \texttt{oct.h}, en él
se encuentran las definiciones de todos los argumentos necesarios.

Empezaremos con el siguiente archivo de ejemplo que llamaremos
\texttt{eqlorentz.cc}:

\begin{verbatim}
#include <octave/oct.h>
DEFUN_DLD (eqlorentz,args, ,
    "Ecuacion de Lorentz en C++")
    {
    ColumnVector xdot (3);
    ColumnVector x (args(0).vector_value());
    int a=10;
    int b=28;
    double c=8./3;
    xdot(0) = a*(x(1)-x(0));
    xdot(1) = x(0)*(b-x(2))-x(1);
    xdot(2) = x(0)*x(1)-c*x(2);
    
    return octave_value (xdot);
    }
\end{verbatim}
Esta es la función de la ecuación diferencial del atractor de Lorentz:
$$
\begin{array}{l}
  \dot{x}=a(y-x)\\
  \dot{y}=x(b-z)-y\\
  \dot{z}=xy-cz\end{array}\qquad con\qquad 
a=10,\,\, b=28,\,\, c=\frac{8}{3}$$

Encontraremos el script que resuelve el problema en el ejercicio
\ref{sec:Ejercicio ode}.  Tenemos que evaluar la ecuación del atractor
un mínimo de 5000 veces, nos ahorraremos algo de tiempo escribiendo la
función en C++ y convirtiéndola en una función de Octave con:

\begin{verbatim}
bash ~/> mkoctfile eqlorentz.cc
\end{verbatim}

El resultado será un archivo llamado \texttt{eqlorentz.oct} que es
equivalente a cualquier archivo de función:

\begin{verbatim}
>> help eqlorentz
eqlorentz is the dynamically-linked function from the file
/home/guillem/sync/CursoScripting/ejercicios/eqlorentz.oct
Ecuacion de Lorentz en C++
>> eqlorentz([1,1,1],1)
ans =
    0.00000
   26.00000
   -1.66667
\end{verbatim}

Otra posibilidad interesante es la de crear subrutinas optimizadas
para un tipo determinado de hardware. No es lo mismo llamar a una
subrutina que funcionaría en cualquier PC que optimizarla para
utilizar todas las posibilidades de un Xeon o de un Opteron.
\texttt{mkoctfile} no es más que un interface a los compiladores de la
família gcc, podemos pasarle parámetros de compilación para conseguir
una optimización máxima.

\subsection{Llamar funciones desde C++}

A medida que nos introduzcamos en la programación de extensiones para
Octave es probable que tengamos la necesidad de tomar funciones como
argumentos.  Un posible ejemplo de ello sería una rutina de integración
de una EDO; sin la función es imposible integrar nada.

Llamar una función desde Octave no es difícil pero tampoco es trivial.
Disponemos del tipo \texttt{octave\_function} que contiene todos
los métodos necesarios, el problema (si es que lo es) es que debe
referirse al argumento como un puntero.  Un \texttt{octave\_function} va a ser
siempre un puntero a una función.  Lo demás es ser un poco congruente
con la definición hecha anteriormente; una función se llama
mediante el método \texttt{do\_multi\_index\_op} que recibe como argumentos
el número de variables de entrada y un \texttt{octave\_value\_list} con
los mismos.  La salida será también un \texttt{octave\_value\_list}.

Ya en octave, para pasar las funciones debemos utilizar un function
handle, una función anónima o la función \texttt{inline}, de otro modo
Octave no sabrá que uno de los argumentos es una función.  Para que
la propia rutina lo compruebe se utilizan los métodos 
\texttt{is\_function\_handle} y \texttt{is\_inline\_function}.

Para ilustrar todo lo anterior utilizaremos el siguiente ejemplo:

\begin{verbatim}
/*testfh.cc*/
# include <octave/oct.h>

DEFUN_DLD(testfh,args, ,
          "testing how C++ can call an octave function")
{
  octave_value_list tmp;
  octave_value_list inval;
  octave_function *input_fcn=0;
  if (args(0).is_function_handle() || args(0).is_inline_function())
    input_fcn = args(0).function_value();
  else
    {
      error("this is not a function handle nor an inline function");
      return octave_value(-1);
    }
  double x = args(1).double_value();
  inval.append(octave_value(x));
  tmp = input_fcn->do_multi_index_op(1,inval);
  return tmp;
}
\end{verbatim}

Esta función recibe un function handle o un inline y un escalar de doble
precisión. Retorna la función evaluada en el punto definido por el escalar.
Para compilarlo haremos:

\begin{verbatim}
bash~/> mkoctfile testfh.cc
\end{verbatim}

Ya dentro de Octave ensayaremos su funcionamiento:

\begin{verbatim}
>> testfh('sin',.123)
error: this is not a function handle nor an inline function
>> testfh(@sin,pi/2)
ans = 1
>> testfh(@(x) sin(x)*exp(x/2),pi/4)
ans = 1.0472
>> testfh(inline('sin(x)*exp(x/2)'),pi/4)
ans = 1.0472
\end{verbatim}

Como vemos es capaz de entender function handles, function handles y
funciones inline.  Esta capacidad abre significativamente el abanico
de posibilidades de octave.  Las funciones escritas en C++ ya no son
algo monolítico, es decir, una subrutina tonta que sólo es capaz de
recibir y devolver valores.  Ahora podemos interactuar con toda
la plataforma de octave, incluso con las funciones pasadas por
cabecera.

\section{Extender Octave con Fortran}

Fortran es el lenguaje de cálculo científico por excelencia. Se ha
abandonado paulatinamente durante los últimos años por el hecho de que
no había ningún compilador libre de Fortran 95. Este hueco ha sido
llenado por los compiladores G95 y gfortran%
\footnote{gfortran es un fork del proyecto G95 por desavenencias en el
  desarrollo con su creador. gfortran está dentro de gcc mientras que
  G95 es un proyecto a parte.%
}. Ahora cualquier desarrollador puede programar en Fortran sin
necesidad de aceptar licencias restrictivas sobre su código.

Evidentemente el lenguaje de extensión natural de Octave es C++ pero
disponemos de la cabecera \texttt{f77-fcn} para comunicarnos con
subrutinas escritas en Fortran. Siendo Fortran tan popular en el
ámbito científico es lógico que Octave le de un trato especial.
Desgraciadamente la capacidad de integración de Fortran en Octave es
bastante limitada debido sobre todo a las dificultades de comunicación
entre C y Fortran.  Otros lenguajes interpretados cumplen mucho mejor
la labor de comunicarse con sus primos compilados como Python, Java o
Ruby.


\subsection{¿Por qué Fortran?}

Las subrutinas escritas en Fortran no son más rápidas en Octave que
las escritas en C++ sin embargo Fortran tiene otras muchas ventajas.
Fortran es, además de un lenguaje para aplicaciones científicas por
excelencia, el único en el que el uso de memoria es puramente
matricial.  Cuando declaramos un array en Fortran 95 lo hacemos con
rango, no sólo con la cantidad de memoria necesaria. Fortran es
claramente mejor que cualquier otro lenguaje de programación cuando
hay que calcular con variables de gran tamaño. Cuando operamos con
vectores, matrices y tensores a la vez es muy fácil equivocarse en los
índices o {}``pisar'' fuera de la matriz. Fortran reserva memoria de
una manera estricta y no tolera un mal uso de ella.

Otro motivo de peso es la manera radicalmente distinta en la que
Fortran y C tratan los arrays.  Mientras en Fortran el uso de la
memoria está detrás del concepto de array, en C no es más que una
sucesión de posiciones de memoria contiguas.  Cuando uno aprende C sin
ser un experto en ordenadores el concepto de puntero se le hace
tremendamente antinatural; más aún si la experiencia previa es en
Fortran. C es un lenguaje especialmente pequeño, esa es su gran
ventaja; pero no por ello es más simple o genera menos sutilezas.
Veamos por ejemplo los dos programas siguientes en C++.  En ellos se
pretende llamar a la ecuación diferencial de Lorentz y sacar el
resultado por pantalla.

\begin{verbatim}
#include <iostream>
using namespace std;

void lorentz(double t,double *y,double *yp)
{
  const int a=10;
  const int b=28;
  const double c=8./3;
  yp[0]=a*(y[1]-y[0]);
  yp[1]=y[0]*(b-y[2])-y[1];
  yp[2]=y[0]*y[1]-c*y[2];
}

int main(int argc,char *argv[])
{
  double y[3];
  double yp[3];

  y[0]=1.;
  y[1]=1.;
  y[2]=1.;
  lorentz(1.,y,yp);
  int i;
  for (i=0;i<3;i++)
    {
      cout << yp[i] << '\n';
    }
  return 0;
}
\end{verbatim}

La salida por pantalla es la siguiente:
\begin{verbatim}
guillem@desktop:~$ ./a.out
0
26
-1.66667
\end{verbatim}

Que es precisamente el resultado de la ecuación diferencial. Llama la
atención el hecho que en la cabecera no hemos definido ningun array
sino punteros.  En la cabecera de la función las definiciones

\begin{verbatim}
*double
double[3]
double[]
\end{verbatim}
Son equivalentes.

Ahora pensamos que a lo mejor C++ es muy listo y que podemos
ahorrarnos el bucle para obtener el resultado.  Probamos lo siguiente:

\begin{verbatim}
#include <iostream>
using namespace std;

void lorentz(double t,double *y,double *yp)
{
  const int a=10;
  const int b=28;
  const double c=8./3;
  yp[0]=a*(y[1]-y[0]);
  yp[1]=y[0]*(b-y[2])-y[1];
  yp[2]=y[0]*y[1]-c*y[2];
}

int main(int argc,char *argv[])
{
  double y[3];
  double yp[3];

  y[0]=1.;
  y[1]=1.;
  y[2]=1.;
  lorentz(1.,y,yp);
  int i;
  cout << yp << '\n';
  
  return 0;
}
\end{verbatim}

Y al ejecutar el programa:
\begin{verbatim}
0xbfce1360
\end{verbatim}
El programa ha escupido una dirección de memoria.  Para un programador
experimentado en C esto resultará tan obvio como desconcertante para
el resto de los mortales.  La relación entre los arrays y los punteros
es de amor-odio.  Salpica cualquier programa escrito en C o en C++ y
puede llegar a ser una tortura cuando sólo nos interesan los arrays.
Fortran no sólo es más seguro sino que ayuda a programar con pocos
errores tanto en la implementación del algoritmo como en el uso de la
memoria.

Otra ventaja importante es que la biblioteca estándar de Fortran
contiene funciones orientadas al cálculo científico y que la mayoría
de las bibliotecas de cálculo numérico están pensadas para comunicarse
con Fortran y no con C o C++.


\subsection{La difícil comunicación entre C y Fortran}
\label{sec:comunica-c-fort}
Fortran y C son lenguajes diametralmente opuestos, sobretodo en el uso
de la memoria. Mientras C permite al usuario jugar con las direcciones
de memoria de un modo transparente, incluso accediendo explícitamente
a los registros; Fortran utiliza la memoria en sentido estricto y
dejando pocas libertades al usuario. La gran diferencia entre ambos es
sin duda las llamadas a funciones y subrutinas, C llama por valor y
Fortran por referencia. ¿Qué significa eso? No olvidemos que Matlab
está escrito en C y Octave en C++, ambos utilizan una infraestructura
equivalente a cualquier programa en C. Para entender bien qué es una
llamada por valor y una llamada por referencia es necesario que
entendamos qué es una dirección de memoria y qué es un puntero.

Los ordenadores tienen tres tipos de memoria: la memoria cache, la
memoria física y la memoria virtual. La más importante de ellas para
un programa es la memoria física y es en la que debemos pensar cuando
programamos. Entrar en más disquisiciones sobre arquitectura de
ordenadores ahora sería inútil. Los programas no son más que procesos
de gestión de memoria; es almacenar algo en un sitio y decir qué hay
que hacer con ello. Si queremos sumar 2 más 2 tendremos que almacenar
un 2 en una posición de memoria, un 2 en otra, pasarlas por un sumador
y luego almacenar el resultado. Los lenguajes de programación de alto
nivel permiten escribir programas sin tener que pensar en todas estas
sutilezas.

Cuando en un código asignamos una variable a un argumento:

\begin{verbatim}
a=2
\end{verbatim}

estamos haciendo dos cosas: primero reservamos el espacio en memoria
necesario para almacenar un 2 y luego le estamos asignando el nombre
de una variable. El nombre está sólidamente unido a la posición de
memoria. Ahora supongamos que queremos referirnos a la variable a.
Podemos hacerlo de dos maneras. La primera es llamar el contenido de
a, es decir, 2. La segunda es referirnos a la variable según la
posición de memoria que ocupa. Llamamos posición de memoria a el punto
de la memoria física en el que se encuentra \emph{realmente} contenida
la variable y se expresa mediante una \emph{dirección de memoria} que
suele ser una cadena de 8 caracteres.

Esto nos permite distinguir entre dos tipos esenciales de variables,
las que contienen un valor y las que apuntan a una determinada
dirección de memoria. Las primeras se llaman variables y las segundas
reciben el nombre de \emph{punteros}. Los programadores experimentados
en C y C++ dominan el concepto de puntero pero los que no lo hayan
oído en su vida lo entenderán mejor con este pequeño programa en C:

\begin{verbatim}
#include <stdio.h>
int a;
int *ptr;
int main(int argc, char *argv[])
{
  a=2;
  ptr =  &a;
  printf("a es %i \n",a);
  printf("la direccion de a es %p \n",ptr);
  printf("el puntero ptr contiene %i \n",*ptr);
  return 0;
}
\end{verbatim}

Primero declaramos una variable de tipo entero y luego un puntero del
mismo tipo, luego le asignamos a la variable el valor de 2 y el
puntero a la posición de memoria de la variable. Imprimimos en
pantalla el resultado para saber qué contiene cada una de ellas y el
resultado es el siguiente:

\begin{verbatim}
a es 2
la direccion de a es 0x403020
el puntero ptr contiene 2
\end{verbatim}

Una vez hemos entendido la diferencia entre una posición de memoria y
el valor que esta contiene daremos el siguiente paso. Supongamos que
acabamos de escribir una función o una subrutina. Esta unidad de
programa necesita argumentos adicionales para funcionar especificados
en su cabecera. Si la función no recibe los argumentos necesarios en
tiempo de ejecución recibiremos un error. Pero acabamos de ver que
para el ordenador es equivalente recibir un valor que la posición de
memoria del bloque que la contiene. En un programa parecido al
anterior definimos la una función suma:

\begin{verbatim}
#include <stdio.h>

int a,b;
int sum(int a,int b)
{ 
  int result;
  result=a+b;
  return result;
}

int main(int argc, char *argv[])
{
  a=2;
  b=3;
  int resultado=sum(a,b);
  printf("la suma de %i y %i es %i \n",a,b,resultado);
  return 0;

}
\end{verbatim}

¿Qué hace C para pasar los valores a la función? C pasa los argumentos
a las funciones por valor. Esto significa que las memoria utilizada
por la función y por el programa principal son independientes. Cuando
pasamos los argumentos a la función suma se copia su valor a las
variables locales sin tocar las globales; la función espera un valor,
no una dirección de memoria; espera un 2, no un 0x748361.

Fortran es diametralmente opuesto a C en este sentido. Todas las
variables de Fortran son en realidad punteros. Fortran no sabe de
valores reales, sólo está interesado en posiciones de memoria. La
identificación entre la memoria y la variable es absoluta. Vamos a
descifrar lo dicho hasta ahora mediante un programa ejemplo:

\begin{verbatim}
program test
integer :: a,b,c
a=2
b=3
call sum(a,b,c)
write(*,*) c
end program test


subroutine sum(arg1,arg2,resultado)
  integer, intent(in) :: arg1,arg2
  integer, intent(out) :: resultado
  resultado=arg1+arg2
end subroutine sum
\end{verbatim}

Cuando llamamos a la subrutina sum no estamos pasando los valores 2 y
3 sino que estamos identificando las posiciones de memoria. Le estamos
diciendo a la subrutina que su arg1 se encuentra en la posición de a y
arg2 en la de b. Por eso tenemos que declarar las variables en las
subrutinas con el atributo intent; estamos operando con las mismas
posiciones de memoria, si nos equivocamos no podemos volver atrás.
Esto se llama pasar los argumentos por \emph{referencia}. Este punto
de vista es mucho menos versatil pero evita el uso de más memoria de
la necesaria. Por eso Fortran es un lenguaje muy indicado para manejar
variables de gran tamaño.

Esto significa que programando en C tendremos que pasar punteros y no
variables a las funciones escritas en Fortran.


\subsection{Llamar una función de C desde Fortran o la manera más difícil de sumar 2+2}

Para entender mejor el significado de la llamada por referencia en
Fortran vamos a aprovechar que los archivos objeto creados con el GCC
son independientes del frontend utilizado.  La estrategia es la
siguiente, crearemos una función en C que sume la primera y la segunda
variable y retorne el resultado en la tercera.  El código es el
siguiente:
\begin{verbatim}
void c_function_(int *a,int *b,int *c)
{
  *c=*a+*b;
}
\end{verbatim}
Algunas observaciones importantes:
\begin{enumerate}
\item Las funciones del tipo \texttt{void} son equivalentes a las
  subrutinas en Fortran.
\item El nombre de la función va seguido de una barra baja (trailing
  underscore) para que el nombre de la subrutina generada sea
  compatible con Fortran.  Ignoro el por qué pero de otro modo no
  funciona.
\item Como la función espera una asignación de argumentos por
  referencia espera punteros a las variables.  En la función debemos
  ser consecuentes y utilizar el operador de retorno de referencia
  para hacer la operación.
\end{enumerate}
Compilamos la función con:
\begin{verbatim}
bash$~/> gcc -c c_function.c
\end{verbatim}

Sólo falta crear el programa que implemente la llamada.  Es el
siguente:
\begin{verbatim}
program fortran_c_call

  integer :: a,b,c

  a=2
  b=2

  call c_function(a,b,c)

  write(*,*) c
end program fortran_c_call
\end{verbatim}
Lo compilamos y lo ejecutamos:
\begin{verbatim}
bash$~/> gfortran fortran_c_call.f90 c_function.o
bash$~/> a.out
        4
\end{verbatim}
Esta operación es imposible si el código objeto generado es
incompatible. Por muy extraño que parezca este tipo de
interoperabilidad es típica de un sistema UNIX.  Aunque C y Fortran
puedan parecer a veces agua y aceite no debemos olvidar que los
compiladores de Fortran no son más que programas escritos en C.

\subsection{Punteros y arrays}

El tipo esencial de Fortran es el array. Un array es un bloque de
memoria que contiene elementos del mismo tipo. Definimos un array
siempre con el apellido de su tipo; hablamos de array de enteros,
array de reales de doble precisión... En las variantes de Fortran más
recientes los arrays tienen rango. Fortran no trata del mismo modo un
array de rango 2 que uno de rango 3, ordenará los elementos de un modo
distinto para optimizar la lectura y la escritura. Hemos dicho que
todas las variables son en realidad punteros; un array es un puntero
al primer elemento, una sentencia de reserva de la memoria necesaria y
una regla de lectura y escritura. Si intentamos leer un elemento fuera
de los límites del array Fortran nos dará un error en tiempo de
ejecución

C no tiene ningún método para definir arrays, C sólo sabe reservar
memoria. Si intentamos emular el comportamiento de Fortran debemos
tener mucha precaución. Por mucho que intentemos reservar la memoria
justa y no llamar ningún elemento que no exista dentro de la variable
el rango desaparece. Para C estas dos declaraciones:

\begin{verbatim}
array[15];
array[3][5];
\end{verbatim}
son equivalentes. Será muy importante utilizar las cabeceras de Octave
porque nos proporcionarán los tipos necesarios para utilizar
cómodamente matrices y arrays.


\subsection{Escritura de \emph{wrappers} para funciones en
  Fortran.}

Aunque este es un conveniente importante no se requieren grandes
conocimientos de C++ para escribir uno. Para escribir un wrapper a la
ecuación diferencial del atractor de Lorentz:

\begin{verbatim}
subroutine eqlorentzfsub(x,xdot)
  real(8),dimension(:),intent(in) :: x
  real(8),dimension(:),intent(out) :: xdot
  real(8) :: a=10d0,b=28d0
  real(8) :: c
  c=8d0/3d0
  xdot(1)= a*(x(2)-x(1))
  xdot(2)= x(1)*(b-x(3)-x(2))
  xdot(3)= x(1)*x(2)-c*x(3)
end subroutine eqlorentzfsub
\end{verbatim}
El wrapper correspondiente para esta función es el siguiente:

\begin{verbatim}
#include <octave/oct.h>
#include "f77-fcn.h"
extern "C" int F77_FUNC (eqlorentzfsub,
                           EQLORENTZFSUB)(double*,double*);
DEFUN_DLD(eqlorentzf,args, ,"xdot=eqlorentz(x)")
{
  octave_value_list retval;
  ColumnVector wargin(args(0).vector_value());
  ColumnVector wargout(3);
  F77_FUNC(eqlorentzfsub,EQLORENTZFSUB)(wargin.fortran_vec(),
                                        wargout.fortran_vec());
  retval.append(wargout);
  return retval;
}
\end{verbatim}
Hacemos siempre uso de los objetos propocionados por las cabeceras de
octave, para ello se supone que tenemos nociones de programación
orientada a objetos. El wrapper sirve únicamente para pasar las
variables de entrada y salida a punteros que Fortran sea capaz de
reconocer.  Las variables de entrada y salida de la función para
octave van a ser \texttt{x} y \texttt{xdot} y las someteremos siempre
a las mismas transformaciones hasta ser punteros:

\begin{enumerate}
\item Recogeremos la variable de entrada de la función que Octave
  dejará en la variable \texttt{args}. Si es un escalar podemos
  utilizar una variable propia de C, si queremos un vector o una
  matriz tendremos que utilizar un tipo propio de Octave. En el caso
  del wrapper anterior declaramos un \texttt{ColumnVector} llamado
  \texttt{wargin} inicializado con la variable \texttt{args(0)} en
  forma de vector.
\item Declararemos la variable de salida, en este caso otro
  \texttt{ColumnVector}, al que también asignaremos un puntero.
  También declararemos la variable de retorno de la función de Octave,
  siempre como \texttt{octave\_value\_list}.
\item Llamaremos la función previamente compilada en Fortran mediante
  \texttt{F77\_FUNC} a la que pasaremos como argumentos los punteros
  que hemos creado para ella, en nuestro caso \texttt{argin} y
  \texttt{argout}. Previamente y fuera de la función
  \texttt{DEFUN\_DLD} declararemos la función externa donde
  definiremos \textbf{todos los argumentos como punteros a las
    variables de entrada y salida}.
\item Pasaremos la variable de salida \texttt{wargout} a la variable
  \texttt{retval} de retorno de la función en Fortran.
\end{enumerate}
Una vez tengamos la función y el wrapper escritos compilaremos la
función en Fortran con \textbf{un compilador compatible con el que se
  ha compilado octave}. El wrapper llamará al código objeto que se
haya creado con el archivo de Fortran. Los comandos son los
siguientes:

  \begin{verbatim}
bash$~/> gfortran -c eqlorentzfsub.f

bash$~/> mkoctfile eqlorentzf.cc eqlorentzfsub.o
\end{verbatim}
Una vez realizado el proceso aparecerá un archivo llamado
\texttt{eqlorentzf.oct} que ya es una función que Octave es capaz de
entender. La llamaremos en Octave del modo usual:

\begin{verbatim}
>> eqlorentzf([1,2,3])
ans =
   10
   23
   -6
\end{verbatim}
La necesidad de que los dos compiladores sean compatibles es un gran
contratiempo. De momento Octave sólo se puede compilar con la versión
3 del compilador GCC que sólo es capaz de compilar código en Fortran
77. Si intentamos mezclar archivos compilados con gcc 4, que entiende
Fortran 95, y gcc 3 muy probablemente tendremos un Segmentation Fault.
De momento nos conformaremos con escribir código en Fortran 77, aunque
no por mucho tiempo. Sólo como ejemplo la misma función en Fortran 77
se escribiría como:

\begin{verbatim}
      subroutine eqlorentzfsub(x,xdot)
      real*8 x(*),xdot(*)
      real*8 a,b,c

      a=10d0
      b=28d0
      c=8d0/3d0

      xdot(1)= a*(x(2)-x(1))
      xdot(2)= x(1)*(b-x(3)-x(2))
      xdot(3)= x(1)*x(2)-c*x(3)

      end subroutine eqlorentzfsub
\end{verbatim}
Podemos escribir wrappers que sean capaces de retornar los mensajes de
error de la subrutina. Para ello es necesario que el wrapper llame a
la función \texttt{F77\_XFCN} en vez de \texttt{F77\_FUNC}. Por
ejemplo, un wrapper para la subrutina siguiente%
\footnote{La subrutina está escrita en Fortran 77. Aunque se considere
  una variante obsoleta de Fortran es necesario conocer sus
  particularidades porque es el lenguaje en el que están escritas la
  mayoría de las bibliotecas de funciones. Es por ello que Octave
  prefiere utilizar una nomenclatura orientada a Fortran 77. Esto no
  genera ningún problema porque el código objeto generado por los
  compiladores de Fortran 77 es compatible con los de Fortran 95.
  Wrappers más sofisticados como f2py ya incorporan una sintaxis
  parecida a Fortran 95.%
}:
\begin{verbatim}
      SUBROUTINE TNINE (IOPT, PARMOD, PS, X, Y, Z, BX, BY, BZ)
      INTEGER IOPT
      DOUBLE PRECISION PARMOD(10), PS, X, Y, Z, BX, BY, BZ    
C     Este es un ejemplo para probar un wrapper.
C     Asigna la suma de PARMOD a PS, y X, Y, Z a BX, BY, BZ
      INTEGER I
      PS = 0D0
      DO I=1, 10
         PS = PS + PARMOD (I)
      END DO
      BX = X
      BY = Y
      BZ = Z
      END
\end{verbatim}
sería:

\begin{verbatim}
#include <octave/oct.h>
#include "f77-fcn.h"
extern "C"
{
  int F77_FUNC (tnine, TNINE) (const int & IOPT, const double* PARMOD,
                               double & PS,
                               const double & X, const double & Y,
                               const double & Z,
                               double & BX, double & BY, double & BZ );
}
DEFUN_DLD (t96, args, ,
           "- Loadable Function: [PS, BX, BY, BZ] = t96 (PM, X, Y, Z) \n\
 \n\
Returns the sum of PM in PS and X, Y, and Z in BX, BY, and BZ.")
{
  octave_value_list retval;
  const int dummy_integer = 0;
  Matrix pm;
  const double x = args(1).double_value(), y = args(2).double_value(),
  z = args(3).double_value();
  double ps, bx, by, bz;
  pm = args(0).matrix_value ();
  F77_XFCN (tnine, TNINE,
           (dummy_integer, pm.fortran_vec(), ps, x, y, z, bx, by, bz) );
  if (f77_exception_encountered)
    {
      error ("unrecoverable error in t96");
      return retval;
    }
  retval(0) = octave_value (ps);
  retval(1) = octave_value (bx);
  retval(2) = octave_value (by);
  retval(3) = octave_value (bz);
  return retval;
}
\end{verbatim}
Para convertir el archivo y el wrapper en una función de Octave
introducimos en la consola:

\begin{verbatim}
bash $~/> mkoctfile t96.cc tnine.f
\end{verbatim}
Y ya podemos llamarla como cualquier otra función de la colección:

\begin{verbatim}
>> help t96
t96 is the dynamically-linked function from the file
/home/guillem/Desktop/sandbox/t96.oct
- Funcion: [PS, BX, BY, BZ] = t96 (PM, X, Y, Z)
Asigna la suma de PM a PS y  X, Y, Z a BX, BY, BZ.
>> [uno,dos,tres,cuatro]=t96(1:10,pi,e,4)
uno = 55
dos = 3.1416
tres = 2.7183
cuatro = 4
\end{verbatim}

Hemos declarado todas las variables de la subrutina, tanto las
internas como las externas, para que si se produce un error en la
parte escrita en Fortran Octave sea capaz de encontrarlo. El wrapper
se encarga de tomar los argumentos de entrada, convertirlos en algo
que Fortran sea capaz de entender y recoger el resultado.

Octave también soporta la definición de los tipos derivados de
Fortran.  Para la subrutina:

\begin{verbatim}
subroutine f95sub2(din,dout)
type              ::        mytype
  real*8          ::        mydouble
  integer*4       ::        myint
end type mytype
type (mytype)     ::        din , dout
dout%mydouble = din%mydouble ** float( din%myint )
dout%myint = din%myint * din%myint
end subroutine f95sub2
\end{verbatim}

Primero la compilaremos con

\begin{verbatim}
bash $ ~/> gfortran -c f95sub.f90
\end{verbatim}

Escribiremos el wrapper:
\begin{verbatim}
#include <octave/oct.h>
#include <octave/f77-fcn.h>
struct mystruct {
  double mydouble;
  int myint;
  };
extern "C" int F77_FUNC (f95sub,f95SUB) ( mystruct &, mystruct &); 
DEFUN_DLD (f95demo, args , ,"[w,z] = f95demo( x , y ) \
returns w  = x ^y and z = y * y for integer y") {
  octave_value_list retval;
  mystruct dinptr , doutptr;
  dinptr.mydouble = args(0).scalar_value();
  dinptr.myint = int( args(1).scalar_value() );
  F77_FUNC (f95sub,f95SUB) (dinptr,doutptr );
  retval.append( octave_value( doutptr.mydouble ) );
  retval.append( octave_value( double( doutptr.myint  ) ) );
  return retval;
}
\end{verbatim}
Y finalmente llamaremos a \texttt{mkoctfile}:
\begin{verbatim}
bash$~/>mkoctfile f95demo.cc f95sub.o
\end{verbatim}

\section{Extender C++ con Octave}

Octave es un proyecto de software libre con un diseño bastante
acertado.  Tanto es así que las librerías de extensón de Octave junto
con la capacidad de incrustar el intérprete en una aplicación hace que
Octave sea una biblioteca de lo más completa. Esto proporciona a
Octave una cualidad no tan esperada; \textbf{puede utilizarse como
  extensión natural de C++ para el cálculo científico}. La cabecera
oct.h proporciona todos los tipos necesaros para el cálculo numérico,
matrices, vectores fila y columna, operaciones matriciales y
vectoriales... ¡Incluso resolución de EDOs! Para abrir boca nada mejor
que un pequeño experimento con un programa de lo más simple:

\begin{verbatim}
/*testoctavestandalone.cc*/
#include <iostream>
#include <oct.h>
int main(void)
{
  std::cout << "Octave y c++ \n";
  Matrix a=identity_matrix(2,2);
  std::cout << a;
}
\end{verbatim}

Para compilarlo utilizaremos la aplicación \texttt{mkoctfile} como
viene siendo habitual pero con la opción \texttt{-{}-link-stand-alone}
para generar un ejecutable.

\begin{verbatim}
$ mkoctfile --link-stand-alone testoctavestandalone.cc -o test.exe
\end{verbatim}
Y lo ejecutaremos, en un entorno UNIX...

\begin{verbatim}
$ ./test.exe
Octave y c++
 1 0
 0 1
\end{verbatim}

La biblioteca en C++ de Octave es mucho más potente como demuestra el
siguiente ejemplo:

\begin{verbatim}
/*test2.cc*/
#include <iostream>
#include <oct.h>
int main(void)
{
  Matrix a=Matrix(2,2);
  ColumnVector b=ColumnVector(2);
  a(0,0)=2.;
  a(1,0)=5.;
  a(0,1)=-6.;
  a(1,1)=3.;
  b(0)=1.;
  b(1)=0.;
  std::cout << a.solve(b);
  return 0;
}
\end{verbatim}
donde resolvemos el siguiente sistema de ecuaciones:

$$
\left(\begin{array}{cc}
    2 & -6\\
    5 & 3\end{array}\right)x= \left(\begin{array}{c}
    1\\
    0\end{array}\right)$$


\begin{verbatim}
$ mkoctfile --link-stand-alone test2.cc -o test.exe
$ ./test.exe 
0.0833333 
-0.138889 
\end{verbatim}

\section{MEX  (+)}

Si Octave se extiende mediante archivos \texttt{oct} Matlab lo hace
mediante archivos \texttt{mex}. Los dos métodos de extensión son
completamente distintos pero Octave también soporta las extensiones de
Matlab.

De este modo Octave no consigue la compatibilidad a nivel de
intérprete sino que también algunas extensiones de Matlab compilarán
con la biblioteca de Octave sin ningún cambio. Parece entonces
justificado describir sin demasiada precisión las funciones
\texttt{mex}


\subsection{El tipo mxArray}

Octave, al estar escrito en C++, es más estricto en el uso de
memoria. Difícilmente dentro del código escrito en C++ vamos a
utilizar arrays, siempre optaremos por los tipos proporcionados por la
librería de Octave para C++.

Escribir extensiones para Matlab es más simple en ese sentido, los
argumentos de entrada y de salida no son más que un array de punteros
a \texttt{mxArray}.  Esto simplifica el código a escribir pero puede
generar errores de segmentación porque no se ejecuta ningún control de
tipos. Por si aún no ha quedado del todo claro, es significativo que
en Octave sólo se pueda retornar un \texttt{octave_value} o un
\texttt{octave_value_list}.

Si se está ampliando el intérprete de Octave los programadores en C++
preferirán los archivos \texttt{oct} y los programadores en C los
\texttt{mex}. En el caso de Matlab no hay discusión posible.

Como suele suceder en los lenguajes interpretados, el tipo estándar,
en este caso \texttt{mxArray}, es en realidad una estructura C de la
que no es importante conocer demasiados detalles.  El modo usual de
crear, introducir y extraer elementos de la estructura es mediante las
funciones de la propia API; entonces el conocimiento y la experiencia
de la misma es lo que permite escribir mejores extensiones.

\subsection{Un ejemplo que simplemente funciona.}

Para comprobar que realmente Octave consigue este grado de
compatibilidad basta con tomar uno de los ejemplos de archivo mex
dentro de la documentación del propio Matlab.  En él ya se intuyen las
características más importantes de la programación en C para
Matlab. Es el siguiente:

\begin{verbatim}
nclude <string.h>
#include "mex.h"

void mexFunction(int nlhs, mxArray *plhs[ ],int nrhs, const mxArray *prhs[ ])
{
int j;
double *output;
double data[] = {1.0, 2.1, 3.0};


/* Create the array */
plhs[0] = mxCreateDoubleMatrix(1,3,mxREAL);
output = mxGetPr(plhs[0]);


/* Populate the output */
memcpy(output, data, 3*sizeof(double));

}
\end{verbatim}

Es un programa muy sencillo.  No recibe ningún argumento y se limita a
copiar un array de tres elementos en el argumento de
salida. Para compilarla con Octave en la consola de sistema:

\begin{verbatim}
$> mex testmex.c
\end{verbatim}

o bien

\begin{verbatim}
$> mkoctfile --mex testmex.c
\end{verbatim}


Esto generará un archivo llamado \texttt{testmex.mex} que es
equivalente a un archivo \texttt{oct}.  Para probarlo, dentro del
intérprete:

\begin{verbatim}
>> testmex()
ans =

   1.0000   2.1000   3.0000
\end{verbatim}

De este primer ejemplo se deduce que la llamada \texttt{mexFunction}
requiere cuatro argumentos:

% use packages: array
\begin{tabular}{ll}
nlhs & Número de argumentos de salida \\ 
plhs & Array de punteros a los argumentos de salida \\ 
nrhs & Número de argumentos de entrada \\ 
prhs & Array de punteros a los argumentos de entrada.
\end{tabular}

Es importante recordar que \emph{los argumentos de entrada no pueden}
modificarse dentro del archivo \texttt{mex} puesto que pertenecen
al intérprete.  Nunca es una buena manera cambiar variables que no
sabemos cómo están representadas en memoria.

También en este ejemplo hemos aprendido las primeras funciones de la
API: \texttt{mxCreateDoubleMatrix} y \texttt{mxGetPtr}; no es
necesario decir muchos sobre qué hacen.  Un detalle que sí puede pasar
inadvertido es que mientras el puntero \texttt{prhs} puede ser un
puntero nulo, \texttt{plhs} tendrá siempre como mínimo un elemento
puesto que el intérprete siempre está esperando como mínimo un valor.
