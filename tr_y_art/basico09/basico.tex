\documentclass[12pt]{beamer}
\usetheme{Pittsburgh}
\usepackage[spanish]{babel}
\usepackage{times}
\usepackage[T1]{fontenc}
\usepackage[utf8]{inputenc}
\usepackage{graphicx}
\usepackage{tikz}
\usepackage{listings}

\usepackage{pgfpages}
\pgfpagesuselayout{2 on 1}[a4paper,border shrink=5mm]

\renewcommand\shorthandsspanish{}
\noextrasspanish

\title{Curso de Matlab.  Nivel Básico}
\author{Guillem Borrell i Nogueras}

\begin{document}

\lstset{language=Matlab,
  backgroundcolor=\color{black!10},
  numbers=left,
  basicstyle=\small\ttfamily,
  keywordstyle=\color{blue},
  extendedchars=true,
  inputencoding=utf8,
  showspaces=false}

\begin{frame}
  \titlepage
\end{frame}

\begin{frame}
  \frametitle{Yo.}
  \begin{itemize}
  \item Guillem Borrell i Nogueras.
  \item Ingeniero Aeronáutico (aunque no me gustan los aviones).
  \item Becario del Grupo de Investigación de Mecánica de Fluidos
    Computacional de la Universidad Politécnica de Madrid.
  \item Consultor Senior de Englobe Technologies.
  \item \textit{I Have Become Comfortably Numb},
    \url{http://torroja.dmt.upm.es/guillem/blog/}
  \item \textit{Introducción Informal a Matlab y Octave},
    \url{http://iimyo.forja.rediris.es/}
  \end{itemize}
\end{frame}

\begin{large}

\begin{frame}
  \frametitle{¿Qué es Matlab?}
  \begin{itemize}
    \item{Un lenguaje de programación}
    \item{Un lenguaje de programación \emph{interpretado}}
    \item{Un lenguaje de programación \emph{interactivo}}
  \end{itemize}
  \begin{center}
    \textbf{Usar Matlab == Programar en Matlab}
  \end{center}
\end{frame}

\begin{frame}
  \frametitle{¿Qué no es Matlab}
  \begin{itemize}
    \item Una hoja de cálculo
    \item Un programa de cálculo simbólico.  Matlab puede hacer
      $\int_0^1 erf(x)\ dx = 0.486$ pero no 
      $\int erf(x)\ dx = x\ erf(x)+\frac{e^{-x^2}}{\sqrt{\pi}}$
    \item La solución a todos nuestros problemas.
  \end{itemize}
\end{frame}


\defverbatim[colored]\testcode{
  \begin{lstlisting}
    >>
  \end{lstlisting}
}

\begin{frame}
  \frametitle{¿Qué significa interpretado?}
  \begin{itemize}
    \item Un intérprete es un programa.
    \item Es como un actor que hace todo lo que le dice un
      \emph{guión}
    \item Muy parecido a la una calculadora.
    \item Es interactivo.
  \end{itemize}
  \testcode
  Os presento a la consola de Matlab
\end{frame}

\begin{frame}
\frametitle{Algunas mentiras}
\begin{itemize}
  \item Para ser ingeniero aeronáutico no es necesario saber
    programar.
  \item Programar es difícil.
  \item Programar \emph{bien} es fácil.
  \item Los ingenieros programan bien
  \item En la vida basta un lenguaje de programación mientras se
    domine.
\end{itemize}
\end{frame}

\begin{frame}
  \frametitle{Un autoengaño}
  \begin{LARGE}
  \begin{center}
    Si en la escuela sólo me dan seis créditos de informática es
    porque no es importante.
  \end{center}
  \end{LARGE}
\end{frame}

\begin{frame}
  \begin{LARGE}
  \begin{center}
    En Arquitectura nadie enseña Autocad.
  \end{center}
  \end{LARGE}
\end{frame}

\begin{frame}
\frametitle{Problema:}
Representar la integral de la función de Bessel

\[ \int_0^y J_{2.5}\ dx \]

con $y \in [1,5]$

\begin{itemize}
\item ¿Cómo se haría en Fortran?
\item ¿Cómo se haría en Excel?
\end{itemize}
\end{frame}

\defverbatim[colored]\testcode{
\begin{lstlisting}
x=linspace(1,5,100);
intbessel=@(y) quad(@(x) besselj(2.5,x),0,y);
for i=1:100
  z(i)=intbessel(x(i));
end
plot(x,z);
\end{lstlisting}
}

\begin{frame}
\frametitle{En Matlab son 6 líneas}
\testcode
No os preocupéis si no entendéis nada.  Esto es Matlab avanzado.
\end{frame}

\begin{frame}
\frametitle{El resultado}
  \begin{figure}[h]
    \centering{}
    \includegraphics[width=8cm, keepaspectratio]{fig/primera.pdf}
  \end{figure}
\end{frame}


\defverbatim[colored]\testcode{
\begin{lstlisting}
>> 2+2
ans =  4
>> mean([1,2,3,4,5,6,7,8,9])
ans =  5
>> abs(3+4i)
ans =  5
\end{lstlisting}
}

\begin{frame}
  \frametitle{¿Una calculadora programable?}
  \testcode
\end{frame}

\begin{frame}
  \frametitle{Todo esto es muy bonito pero...}
  \begin{itemize}
    \item ¿Es una herramienta realmente útil?
    \item ¿Se usa masivamente en la industria?
    \item ¿Por qué?
    \item ¿Cuánto cuesta Matlab?
    \item ¿Es la única solución?
  \end{itemize}
\end{frame}

\begin{frame}
  \frametitle{Octave}
  \begin{itemize}
    \item Implementación libre y gratuita del lenguaje Matlab
    \item \url{http://www.octave.org}
    \item Programa muy utilizado en GNU/Linux
    \item Versiones para Windows y Mac
    \item QtOctave
    \item \emph{Libre y gratuito}
  \end{itemize}
\end{frame}

\begin{frame}
  \frametitle{El lenguaje Matlab}
  \begin{itemize}
    \item Caracteres especiales
    \item Funciones y scripts
    \item Tipos
    \item Variables
    \item Operadores
    \item Sentencias
    \item \emph{Function handles}
  \end{itemize}
\end{frame}

\defverbatim[colored]\testcode{
\begin{lstlisting}
>> % Este comando sera ignorado
>> 'hola' % 'Hola,Matlab!'
ans = hola

>> 'hola';
>> 'hola', 'que tal'
ans = hola
ans = que tal

>> 'hola', ...
'que tal'
ans = hola
ans = que tal
\end{lstlisting}
}

\begin{frame}
\frametitle{Caracteres especiales}
\testcode
\end{frame}

\begin{frame}
\frametitle{El directorio de trabajo}
\begin{itemize}
\item Matlab puede ejecutar archivos con código
\item Matlab puede cargar archivos de datos
\item La biblioteca de funciones está formada por archivos con código.
\item Matlab busca en sus directorios de sistema más el \emph{directorio de
  trabajo}
\item Variable \texttt{path}
\end{itemize}
\end{frame}


\defverbatim[colored]\testcode{
\begin{lstlisting}
function [sal1,sal2,...] = nombre(ent1,ent2,...)
  sentencias ejecutables
  sal1 = ...
  sal2 = ...
\end{lstlisting}
}

\begin{frame}
\frametitle{Funciones. Sintaxis}
\testcode

Lo guardaremos todo en \emph{el directorio de trabajo} en un archivo
llamado \emph{nombre.m}
\end{frame}

\begin{frame}
\frametitle{Scripts}
\begin{itemize}
\item Un script es un programa
\item Un programa es una secuencia de instrucciones ejecutables
\item Un programa no depende de variables externas
\item También se guarda en un archivo \emph{.m} en el \emph{directorio
  de trabajo}
\item Se ejecuta escribiendo el nombre del archivo en la consola o
  pulsando \texttt{F5} en el editor.
\end{itemize}
\end{frame}

\defverbatim[colored]\testcode{
\begin{lstlisting}
function y = aprsin(x)
  y=x-(x.^3)/6
\end{lstlisting}
}
\begin{frame}
  \frametitle{Nuestra primera función}
Abrimos un archivo nuevo en el editor
\testcode
Y lo guardamos en el directorio de trabajo como \texttt{aprsin.m}.
\end{frame}

\defverbatim[colored]\testcode{
\begin{lstlisting}
x=linspace(-pi,pi,100);
for i = 1:100
  y(i)=aprsin(x(i));
end
plot(x,[y;sin(x)])
\end{lstlisting}
}

\begin{frame}
\frametitle{Nuestro primer script}
En un archivo nuevo del editor
\testcode
Lo guardamos con el nombre \texttt{comparar.m} \emph{en el directorio
  de trabajo}. Luego pulsamos \texttt{F5}
\end{frame}

\begin{frame}
\frametitle{El resultado}
  \begin{figure}[h]
    \centering{}
    \includegraphics[width=8cm, keepaspectratio]{fig/comparar.pdf}
  \end{figure}
\end{frame}


\defverbatim[colored]\testcode{
\begin{lstlisting}
help eig
\end{lstlisting}
}

\begin{frame}
\frametitle{Ayuda. Función \texttt{help}}
\begin{itemize}
\item En Matlab todo es una función
\item Cada función contiene una pequeña ayuda
\item Para consultar la ayuda existe la función \texttt{help}
\end{itemize}
\testcode
\end{frame}

\end{large}

\end{document}
