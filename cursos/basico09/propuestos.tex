\documentclass{article}
\usepackage[T1]{fontenc}
\usepackage[utf8]{inputenc}
\usepackage[spanish]{babel}
\usepackage{graphicx}
\usepackage{tikz}
\usepackage{listings}
\usepackage[a4paper]{geometry}

\renewcommand\shorthandsspanish{}
\noextrasspanish

\title{Curso de Matlab.  Nivel Básico. Problemas propuestos}
\author{Guillem Borrell i Nogueras}

\begin{document}

\lstset{language=Matlab,
  backgroundcolor=\color{black!10},
  numbers=left,
  basicstyle=\small\ttfamily,
  keywordstyle=\color{blue},
  extendedchars=true,
  inputencoding=utf8,
  showspaces=false}

\maketitle

\section*{Ejercicio 3}
Dadas las siguientes variables:

\[
A = \left( \begin{array}{ccc}
1&2&3\\
4&5&6\\
7&8&9
\end{array} \right) \quad
b = \left( \begin{array}{c}
1\\2\\3
\end{array} \right) \quad
c = \left( \begin{array}{ccc}
1&2&3
\end{array} \right) 
\]

realizar las siguientes operaciones

\begin{enumerate}
\item $A\cdot b$
\item $\sum_i A_{ij}c_i$
\item $e_{ij}=b_ic_j$
\end{enumerate}

Aplicar en cada resultado la función $x^2 \sin x$

\section*{Ejercicio 4}

La serie de Fibonacci se define mediante la siguiente regla de recurrencia\[
F_{n}=\left\{ \begin{array}{cc}
1 & n=1\\
1 & n=2\\
F_{n-1}+F_{n-2} & n>2\end{array}\right.\]

Escribir dos funciones que devuelvan, respectivamente, en función de $n$

\begin{enumerate}
\item El término $n$ de la sucesión de Fibonacci
\item Un vector con los primeros $n$ términos de la sucesión de
  Fibonacci.
\end{enumerate}

\section*{Ejercicio 5}

Construir una estructura de datos que contenga las funciones
trigonométricas $\sin$, $\cos$, y $\tan$ y llamarlas en el punto
$\pi/2$ a partir de la misma estructura

\section*{Ejercicio 6}

Generar la siguiente matriz

\[
L = \left( \begin{array}{ccccccc}
1&0&0&0&0&0&0\\
1&-2&1&0&0&0&0\\
0&1&-2&1&0&0&0\\
0&0&1&-2&1&0&0\\
0&0&0&1&-2&1&0\\
0&0&0&0&1&-2&1\\
0&0&0&0&0&0&1
\end{array} \right)
\]

usando también la función \texttt{diag}

\section*{Ejercicio 7}

Hacer la integral

\[ I=\int_{-10}^{10}\int_{-10}^{10}e^{-(x^2+y^2)}dx\ dy \]

que tiene como solución una burda aproximación a $\pi$.  Hay que
utilizar una función anónima y hallar el resultado en sólo una línea
de código.

\section*{Ejercicio 8}

Representar en una misma ventana y dos frames (uno superior y otro
inferior) la función

\[ \sqrt{x} \sin(1/x)\ \ x\in[0.001,1] \]

en escala normal y en escala semilogarítmica en el eje x.  La segunda
gráfica, la logarítmica, tiene un problema de resolución cerca de
$x=0$.  ¿Cómo puede arreglarse sin añadir más puntos?

\section*{Ejercicio 9}

Resolver el siguiente problema no stiff

\[ \dot {\vec x} = \left(
  \begin{array}{c}
a(y-x)\\
x(b-z)-y\\
xy-cz  
  \end{array}
\right)   \]

con $a=10$, $b=28$ y $c=8/3$, $t \in [0,50]$ y
$(x_0,y_0,z_0)=(1,1,1)$ y representar la solución en tres dimensiones
con una curva paramétrica mediante la función \texttt{plot3}

\end{document}


