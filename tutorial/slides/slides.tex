\documentclass[12pt]{beamer}
\usetheme{Warsaw}
\usepackage{times}
\usepackage[T1]{fontenc}
\usepackage[utf8]{inputenc}
\usepackage{graphicx}
\usepackage{tikz}
\usepackage{listings}

%\usepackage{pgfpages}
%\pgfpagesuselayout{4 on 1}[a4paper,border shrink=5mm]

\title{Introducción a Matlab y Octave}
%\subtitle{Lesson 0}
\author{Guillem Borrell i Nogueras\\
Laboratorio de Mecánica de Fluidos Computacional}

\begin{document}

\lstset{language=Matlab,
  backgroundcolor=\color{black!10},
  numbers=left,
  basicstyle=\small\ttfamily,
  keywordstyle=\color{blue},
  extendedchars=true,
  inputencoding=utf8,
  showspaces=false}

\begin{frame}
  \titlepage
\end{frame}

\begin{frame}
  \tableofcontents[pausesections]
\end{frame}


\section{Contenidos}

\begin{frame}
\frametitle{Contenidos}
\begin{itemize}
\item Introducción
\end{itemize}
\end{frame}

\begin{frame}
  \frametitle{Objetivos}
  \begin{itemize}
  \item Matlab basic skills.
  \item Understanding symbolic operations.
  \item Visualization.
  \item Simple Real World applications.
  \end{itemize}
\end{frame}

\begin{frame}
  \frametitle{Antes de empezar}
  \begin{Huge}
    \begin{center}
      Matlab is not a symbolic math tool.
    \end{center}
  \end{Huge}
  Matlab will do the weightlifting, we will take care of the rest.
\end{frame}

\begin{frame}
  \frametitle{Matlab is a programming language}
  \begin{itemize}
  \item Designed for Numerical Analysis.
  \item Numeric vs. Symbolic
    \begin{itemize}
    \item Matlab is a mind-numbing powerful calculator.
    \end{itemize}
  \item It is nothing but a programming language.
    \begin{itemize}
    \item Not a spreadsheet
    \item Not a symbolic algebra engine
    \item Not a solution for all your problems.
    \end{itemize}
  \end{itemize}
\end{frame}

\defverbatim[colored]\testcode{
\begin{lstlisting}
f = @(x) quad(@(y) exp(-y.^2),0,y)
\end{lstlisting}
}
\begin{frame}
  \frametitle{¿Mind numbing?}
This is advanced Matlab

\[ f(x) = \int_0^x \exp(-y^2) dy \]

\testcode

BTW: this is not a symbolic operation.
\end{frame}

\begin{frame}
  \frametitle{Why Matlab?}

  \begin{itemize}
  \item Most extended tool in Science and Engineering
  \item De-facto standard for short programs in Engineering
  \item Huge Library of functions
  \item Tool-boxes for almost everything.
  \end{itemize}

  Learning a bit of Matlab is a goal too.
\end{frame}

\section{Fundamentos de Programación}
\subsection{Programming Fundamentals}

\defverbatim[colored]\testcode{
\begin{lstlisting}
a = 1;
\end{lstlisting}
}
\begin{frame}
  \frametitle{Variables and Literals}
\testcode
\begin{itemize}
\item \texttt{a} is a variable
\item \texttt{=} is the assignment operator
\item \texttt{1} is a literal for the scalar 1 in the default
  precision.
\end{itemize}
\begin{center}
  What does all this mean?
\end{center}
\end{frame}

\begin{frame}
  \frametitle{The meaning of all this}
  \begin{itemize}
  \item Variables store
  \item The asignment operator assigns anything to a variable
  \item All the computations are in the right hand side
  \item Computation is any operation involving variables (stored) and
    literals (new).
  \end{itemize}
\end{frame}

\begin{frame}
  \frametitle{Variables}
  \begin{itemize}
  \item Any word can become a variable name.
  \item Unless it uses a reserved character like \texttt{+},
    \texttt{-}, \texttt{*}, \texttt{/}, \texttt{\^}, \texttt{.}, ...
  \item Or is a built in variable like \texttt{ans} or \texttt{pi}
  \item A variable can store anything.
  \end{itemize}
\end{frame}

\begin{frame}
  \frametitle{Literals}
  \begin{itemize}
  \item A literal is any valid Matlab value written by us.
  \item There is a literal for almost every Matlab type: scalars,
    vectors, matrices, functions...
  \item Almost means that there are some types missing, but we won't
    miss a literal for them.
  \item We will use literals for scalars, vectors and functions.
  \end{itemize}
Now we can start coding.
\end{frame}

\subsection{The Matlab IDE}
\begin{frame}
  \frametitle{Tools}
Pieces of the Matlab IDE
\begin{itemize}
\item The Matlab Console
\item The Matlab Editor
\item The Matlab Explorer
\item The Matlab Debugger
\end{itemize}
We will use only the Console and the Editor.
\end{frame}

\begin{frame}
  \frametitle{The Matlab Console}

\pgfimage[width=\linewidth]{gui}

\end{frame}

\begin{frame}
  \frametitle{The Matlab Editor}
\pgfimage[width=\linewidth]{editor}

\end{frame}

\subsection{The Octave Interpreter}
\begin{frame}
  \frametitle{Octave}
  \begin{itemize}
  \item Octave is a free (as in freedom) interpreter for the Matlab
    programming language.
  \item Octave provides only the interpreter and the interactive
    console. We must use an external source code editor.
  \item Covers the 90\% of the language but only the 30\% of the
    toolkits.
  \end{itemize}

\end{frame}

\section{Matlab Fundamentals (Language)}
\subsection{Matlab types and literals}

\defverbatim[colored]\testcode{
\begin{lstlisting}
a = 1.234;
\end{lstlisting}
}
\begin{frame}
  \frametitle{Scalars}
  \begin{itemize}
  \item A scalar is a number.
    \pause
  \item Seriously, it is just a number.
  \item There are many other types that involve numbers...
  \item The decimal point is a dot.
  \end{itemize}
  \testcode
\end{frame}

\end{document}
