\documentclass[12pt]{beamer}
\usetheme{Warsaw}
\usepackage{times}
\usepackage[T1]{fontenc}
\usepackage[utf8]{inputenc}
\usepackage{graphicx}
\usepackage{tikz}
\usepackage{listings}

%\usepackage{pgfpages}
%\pgfpagesuselayout{4 on 1}[a4paper,border shrink=5mm]

\title{Introducción a Matlab y Octave}
%\subtitle{Lesson 0}
\author{Guillem Borrell\\
Lab. de Mecánica de Fluidos Computacional}

\begin{document}

\lstset{language=Matlab,
  backgroundcolor=\color{black!10},
  numbers=left,
  basicstyle=\small\ttfamily,
  keywordstyle=\color{blue},
  extendedchars=true,
  inputencoding=utf8,
  showspaces=false}

\begin{frame}
  \titlepage
\end{frame}

\begin{frame}
  \frametitle{Antes de empezar}
  \begin{itemize}
  \item Guillem Borrell i Nogueras.
  \item  \url{http://iimyo.forja.rediris.es/}
    \begin{itemize}
      \item Introducción Informal a Matlab y Octave
      \item Matemáticas en Ingeniería con Matlab y Octave
      \item Transparencias y ejercicios de este curso
      \item Material de otros cursos
    \end{itemize}
  \end{itemize}
\end{frame}


\begin{frame}
\frametitle{Recordad...}
\begin{center}
Ningún lenguaje se aprende por osmosis
\end{center}
\end{frame}


\begin{frame}
  \tableofcontents[pausesections]
\end{frame}

\section{Introducción}

\begin{frame}
  \frametitle{Objetivos}
  \begin{itemize}
  \item Introducción a la programación.
  \item Aprender un poquito de Matlab
  \item Facilitar vuestro aprendzaje.
  \item Ahorraros futuros disgustos.
  \end{itemize}
\end{frame}

\begin{frame}
  \begin{Huge}
    \begin{center}
      Matlab no es una herramienta de cálculo simbólico.
    \end{center}
  \end{Huge}
  \pause
  Matlab es un lenguaje de programación
\end{frame}

\begin{frame}
  \frametitle{Matlab es un lenguaje de programación}
  \begin{itemize}
  \item Orientado a Cálculo Numérico
  \item Nada de Cálculo Simbólico
    \begin{itemize}
    \item Imaginadlo como una gran calculadora
    \end{itemize}
  \item Si es un lengaje de programación
    \begin{itemize}
    \item No es una hoja de cálculo
    \item No es un juguete
    \item No es la solución a todos nuestros problemas.
    \end{itemize}
  \end{itemize}
\end{frame}

\begin{frame}
  \frametitle{¿Qué es Matlab?}
  \begin{itemize}
    \item{Un lenguaje de programación}
    \item{Un lenguaje de programación \emph{interpretado}}
    \item{Un lenguaje de programación \emph{interactivo}}
  \end{itemize}
  \begin{center}
    \textbf{Usar Matlab == Programar en Matlab}
  \end{center}
\end{frame}


\defverbatim[colored]\testcode{
\begin{lstlisting}
f = @(x) quad(@(y) exp(-y.^2),0,y)
\end{lstlisting}
}
\begin{frame}
  \frametitle{¿Cómo de potente?}
Esto es Matlab avanzado

\[ f(x) = \int_0^x \exp(-y^2) dy \]

\testcode

Por cierto, esto es Cálculo Numérico
\end{frame}

\begin{frame}
  \frametitle{¿Por qué Matlab?}

  \begin{itemize}
  \item Herramienta más extendida en Ciencia e Ingeniería
  \item Estándar de facto para escribir pequeños programas.
  \item Enorme librería estándar
  \item Toolboxes para casi cualquier disciplina de la Ingeniería
  \end{itemize}

\end{frame}

\section{Fundamentos de Programación}

\defverbatim[colored]\testcode{
\begin{lstlisting}
a = 1;
\end{lstlisting}
}
\begin{frame}
  \frametitle{Variables y literales}
\testcode
\begin{itemize}
\item \texttt{a} es una variable
\item \texttt{=} es el operador asignación
\item \texttt{1} es el literal que significa el número 1
\end{itemize}
\begin{center}
  ¿Qué significa todo esto?
\end{center}
\end{frame}

\begin{frame}
  \frametitle{El significado de todo esto}
  \begin{itemize}
  \item Las variables almacenan en memoria
  \item El operador asignación asigna cualquier cosa a una variable
  \item Todos los cálculos van a la derecha del operador
  \item Los cálculos son cualquier operación entre variables y literales
  \end{itemize}
\end{frame}

\begin{frame}
  \frametitle{Variables}
  \begin{itemize}
  \item Cualquier palabra puede convertirse en variable
  \item Siempre que no use un caracter reservado como \texttt{+},
    \texttt{-}, \texttt{*}, \texttt{/}, \texttt{\^}, \texttt{.}, ...
  \item O no sea un nombre reservado como \texttt{ans} or \texttt{pi}
  \item Una variable puede almacenar cualquier valor.
  \end{itemize}
\end{frame}

\begin{frame}
  \frametitle{Literales}
  \begin{itemize}
  \item Es la manera que tenemos de introducir un valor.
  \item Hay un literal para prácticamente cualquier valor en Matlab:
    escalares, vectores, funciones...
  \item Prácticamente significa que no están todos, pero los que
    faltan no los echaremos de menos.
  \end{itemize}
\pause
Pero antes de programar, algo de evangelismo.
\end{frame}

\subsection{El entorno de desarrollo Matlab}
\begin{frame}
  \frametitle{Herramientas}
Partes del entorno de escritorio Matlab
\begin{itemize}
\item La consola
\item El editor
\item El explorador
\item El debugger
\end{itemize}
Aprenderemos sólo a utilizar la consola y el editor
\end{frame}

\begin{frame}
  \frametitle{La consola}

\pgfimage[width=\linewidth]{../_static/principal_col}

\end{frame}

\begin{frame}
  \frametitle{El editor}
\pgfimage[width=\linewidth]{../_static/editor}
\end{frame}

\begin{frame}
\frametitle{Scripts}
\begin{itemize}
\item Un script es un programa
\item Un programa es una secuencia de instrucciones ejecutables
\item Un programa no depende de variables externas
\item Se guarda en un archivo \emph{.m} en el \emph{directorio
  de trabajo}
\item Se ejecuta escribiendo el nombre del archivo en la consola o
  pulsando \texttt{F5} en el editor.
\end{itemize}
\end{frame}


\defverbatim[colored]\testcode{
\begin{lstlisting}
aprsin = @(x) x - x.^3/6;
x = linspace(-pi,pi,100);
plot(x,aprsin(x),x,sin(x));
\end{lstlisting}
}

\begin{frame}
\frametitle{Nuestro primer script}
En un archivo nuevo del editor
\testcode
Lo guardamos con el nombre \texttt{comparar.m} \emph{en el directorio
  de trabajo}. Luego pulsamos \texttt{F5}
\end{frame}


\begin{frame}
\frametitle{El resultado}
  \begin{figure}[h]
    \centering{}
    \pgfimage[width=8cm]{../_static/aprsin.pdf}
  \end{figure}
\end{frame}

\subsection{El intérprete Octave}
\begin{frame}
  \frametitle{Octave}
  \begin{itemize}
  \item Octave es un intérprete libre y gratuito del lenguaje Matlab
  \item Proporciona el intérprete y la consola interactiva, nosotros
    debemos proporcionar el resto
  \item Proporciona aproximadamente el 90\% del lenguaje y el 30\% de
    los toolkits.
  \end{itemize}
\end{frame}

\section{El lenguaje de programación Matlab}
\subsection{Tipos y literales}

\defverbatim[colored]\testcode{
\begin{lstlisting}
a = 1.234;
\end{lstlisting}
}
\begin{frame}
  \frametitle{Escalar}
  \begin{itemize}
  \item Un escalar es un número.
    \pause
  \item Sí, sólo un número.
  \item Pero a partir del escalar generamos el resto de tipos
  \item La coma decimal es un punto.
  \end{itemize}
  \testcode
\end{frame}

\begin{frame}
  \frametitle{Operaciones escalares}
  \begin{itemize}
  \item Suma. \texttt{+}
  \item Resta, Número negativo. \texttt{-}
  \item Multiplicación. \texttt{.*}
  \item División. \texttt{./}
  \item Potencia. \texttt{.\^}
  \end{itemize}
\end{frame}

\begin{frame}
  \frametitle{funciones escalares}
  \begin{itemize}
  \item \texttt{exp},\texttt{log},\texttt{log10},\texttt{sqrt}...
  \item \texttt{sin},\texttt{cos},\texttt{tan}...
  \item \texttt{asin},\texttt{acos},\texttt{atan}...
  \item \texttt{sinh},\texttt{cosh}...
  \item \texttt{gamma},\texttt{beta},\texttt{erf},\texttt{sinc}...
  \end{itemize}
\end{frame}

\defverbatim[colored]\testcode{
\begin{lstlisting}
>> exp(1.0i * pi) - 1
\end{lstlisting}
}

\begin{frame}
  \frametitle{Operaciones aritméticas}
  \[ exp(i \pi) - 1 = 0  \]
  \testcode
\end{frame}

\begin{frame}
  \frametitle{Otros tipos}
  \begin{itemize}
  \item La unidad imaginaria es \emph{i}, \emph{j}, \emph{I} o \emph{J}
  \item Las cadenas de texto se introducen entre comillas simples
  \item Los tipos lógicos son \emph{true} y \emph{false}.  \emph{true}
    es $\not \equiv 0$ y \emph{false} es $\equiv 0$
  \end{itemize}
\end{frame}

\defverbatim[colored]\testcode{
\begin{lstlisting}
>> % Este comando sera ignorado
>> 'hola' % 'Hola,Matlab!'
ans = hola

>> 'hola';
>> 'hola', 'que tal'
ans = hola
ans = que tal

>> 'hola', ...
'que tal'
ans = hola
ans = que tal
\end{lstlisting}
}

\begin{frame}
\frametitle{Caracteres especiales}
\testcode
\end{frame}


\defverbatim[colored]\testcode{
\begin{lstlisting}
  >> a = pi
  a = 3.1416
  >> a(1)
  ans = 3.1416
  >> a(1,1)
  ans = 3.1416
  >> a(1,1,1)
  ans = 3.1416
\end{lstlisting}
}

\begin{frame}
  \frametitle{Mira qué curioso}
  \testcode
\end{frame}

\begin{frame}
\frametitle{Escribir matrices}
\begin{itemize}
  \item El espacio o la coma separan elementos de la misma fila
  \item El retorno de carro o el punto y coma separa filas
\end{itemize}
\[ \left(
\begin{array}{ccc}
1&2&3\\
4&5&6\\
7&8&9
\end{array} \right)
\]
\end{frame}


\defverbatim[colored]\testcode{
\begin{lstlisting}
M=[1,2,3;4,5,6;7,8,9];
\end{lstlisting}
}

\begin{frame}
\frametitle{Ejercicio 1}
\testcode
Escribir 
\[ \left(
\begin{array}{ccc}
1&2&3\\
4&5&6\\
7&8&9
\end{array} \right)
\]
de otros 3 modos posibles.
\end{frame}

\begin{frame}
\frametitle{Subíndices}
\begin{itemize}
  \item En Matlab el primer índice cuenta elementos en la columna
  \item El segundo índice cuenta elementos en la fila
  \item \emph{Pero} un vector es siempre fila a no ser que se diga lo
    contrario.
  \item El truco es que no nos preocupen las filas y las columnas,
    sólo los índices
\end{itemize}
\[ M_{ij} = M(i,j) \]
\end{frame}

\defverbatim[colored]\testcode{
\begin{lstlisting}
>> v(4)=2
v =

   0   0   0   2

>> w(4,1)=2
w =

   0
   0
   0
   2
\end{lstlisting}
}
\begin{frame}
\frametitle{No hay quien te entienda}
\testcode
\end{frame}

\defverbatim[colored]\testcode{
\begin{lstlisting}
>> v=[1,2,3,4,5];
>> v(4)
ans = 4
\end{lstlisting}
}

\begin{frame}
\frametitle{Un vector...}
\[v = (1,2,3,\textcolor{red}{4},5) \]
\testcode
\end{frame}

\defverbatim[colored]\testcode{
\begin{lstlisting}
>> M=[1,2,3;4,5,6;7,8,9];
>> M(2,3)
ans =  6
\end{lstlisting}
}

\begin{frame}
\frametitle{Una matriz...}
\[ \left(
\begin{array}{ccc}
1&2&3\\
4&5&\textcolor{red}{6}\\
7&8&9
\end{array} \right)
\]
\testcode
\end{frame}


\defverbatim[colored]\testcode{
\begin{lstlisting}
>> M([1,2],[2,3])
ans =

   2   3
   5   6
\end{lstlisting}
}

\begin{frame}
\frametitle{Podemos indexar con vectores}
\[ \left(
\begin{array}{ccc}
1&\textcolor{red}{2}&\textcolor{red}{3}\\
4&\textcolor{red}{5}&\textcolor{red}{6}\\
7&8&9
\end{array} \right)
\]
\testcode
\end{frame}

\defverbatim[colored]\testcode{
\begin{lstlisting}
>> M(2,:)
ans =

   4   5   6
\end{lstlisting}
}

\begin{frame}
\frametitle{O con índices mudos}
\[ \left(
\begin{array}{ccc}
1&2&3\\
\textcolor{red}{4}&\textcolor{red}{5}&\textcolor{red}{6}\\
7&8&9
\end{array} \right)
\]
\testcode
\end{frame}

\defverbatim[colored]\testcode{
\begin{lstlisting}
\end{lstlisting}
}

\defverbatim[colored]\testcode{
\begin{lstlisting}
>> 0:2:10
ans =
   0   2   4   6   8  10

>> 0:5
ans =
  0  1  2  3  4  5
\end{lstlisting}
}


\begin{frame}
\frametitle{Secuencias}
\begin{itemize}
\item Es una abreviatura común para escribir un vector fila
\item La sintaxis es \texttt{inicio:incremento:final}
\end{itemize}
\testcode
\end{frame}

\begin{frame}
\frametitle{Ejercicio 2} 
Crear la matriz siguiente y extraer de ella la submatriz marcada en rojo.
\[ \left( \begin{array}{ccccc}
11&12&13&14&15\\
21&22&23&24&25\\
31&32&\textcolor{red}{33}&\textcolor{red}{34}&\textcolor{red}{35}\\
41&42&\textcolor{red}{43}&\textcolor{red}{44}&\textcolor{red}{45}\\
51&52&\textcolor{red}{53}&\textcolor{red}{54}&\textcolor{red}{55}
\end{array} \right) \]
\end{frame}

\begin{frame}
\frametitle{Otros tipos}
\begin{itemize}
\item La unidad imaginaria es \emph{i}, \emph{j}, \emph{I} o \emph{J}
\item Las cadenas de texto se introducen entre comillas simples
\item Los tipos lógicos son \emph{true} y \emph{false}.  \emph{true}
  es $\not \equiv 0$ y \emph{false} es $\equiv 0$
\end{itemize}
\end{frame}

\begin{frame}
\frametitle{Operadores}
\begin{itemize}
\item Operadores matriciales \texttt{+}, \texttt{-}, \texttt{*},
  \texttt{/}, \texttt{\^}
\item Operadores escalares \texttt{.*}, \texttt{./}, \texttt{.\^}
\item Operadores lógios matriciales \texttt{\&}, \texttt{|}, \texttt{!}
\item Relaciones de comparación \texttt{<}, \texttt{>}, \texttt{==},
  \texttt{<=}, \texttt{>=}, \texttt{!=}
\item Relaciones lógicas \texttt{\&\&}, \texttt{||}
\end{itemize}
\end{frame}

\defverbatim[colored]\testcode{
\begin{lstlisting}
>> a=rand(3,3);
>> a=rand(3,3);b=rand(3,3);
>> a*b
ans =
    1.0297    0.9105    0.3293
    0.9663    0.8267    0.4211
    0.5355    0.4318    0.3279
>> a.*b
ans =
    0.1824    0.3253    0.0563
    0.5500    0.6003    0.1897
    0.0458    0.0017    0.1822
\end{lstlisting}
}

\begin{frame}
\frametitle{El error más común de Matlab}
\testcode
\end{frame}

\defverbatim[colored]\testcode{
\begin{lstlisting}
>> a=[1,2,3;4,5,6;7,8,9];
>> a.^pi
ans =
    1.0000    8.8250   31.5443
   77.8802  156.9925  278.3776
  451.8079  687.2913  995.0416
>> a^pi
ans =
 1.0e+03 *
 0.69 - 0.0004i 0.85 - 0.0001i 1.01 + 0.0002i
 1.57 - 0.0000i 1.93 - 0.0000i 2.29 + 0.0000i
 2.45 + 0.0003i 3.01 + 0.0001i 3.57 - 0.0002i
\end{lstlisting}
}

\begin{frame}
\frametitle{El error más común de Matlab}
\testcode
\end{frame}


\subsection{Control de flujo}

\subsection{Unidades de programa}

\section{Introducción a la Biblioteca de funciones}

\subsection{Álgebra Lineal}
\subsection{R}

\end{document}
