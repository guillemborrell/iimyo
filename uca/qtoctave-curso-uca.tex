
\section{Primeros pasos con la interfaz \qtoctave}
\label{sec:qtoctave}

A pesar de que la consola de Octave tiene características bastante
avanzadas\footnote{Utiliza la biblioteca \emph{GNU readline}, ofrece
  la posibilidad de acceder a las líneas de órdenes anteriores
  mediante teclas \tecla{flecha arriba}/\tecla{flecha abajo},
  completar los comandos con la tecla \tecla{tabulador}, etc} y
permite acceder a toda la funcionalidad de este lenguaje, muchos
usuarios preferirán emplear una interfaz gráfica que, por ejemplo
facilite editar y almacenar en el disco series de órdenes o programas
completos, o bien acceder a determinadas funcionalidades mediante
menús o barras de herramientas.

Para ello, en el mundo del software libre, existen distintas
posibilidades. Por ejemplo, el potente editor \emacs dispone de un
modo específico que puede significar una opción muy sugerente para
interaccionar con GNU Octave. Ahora bien, en las siguientes secciones,
nos centraremos en una de las interfaces gráficas de usuario que, en la
actualidad, ofrece prestaciones más interesantes para un usuario medio:
\qtoctave.

Este programa recibe su nombre de las bibliotecas QT, desarrolladas
con la finalidad de facilitar la creación de interfaces gráficas.
 
\subsection*{Instalación}

\qtoctave está disponible:
\begin{itemize}
\item A través de su página web, \url{http://qtoctave.wordpress.com/}.
  Esta página es el centro de
  comunicación con el autor y la comunidad de usuarios de \qtoctave.
\item En la forja Forja de RedIris
  (CICA). \url{https://forja.rediris.es/projects/csl-qtoctave/} Esta
  forja es es el centro de desarrollo de
  qtOctave. Para descargas, pulsar botón ``\textit{Ficheros}''
\item En las páginas web de la universidad de Cádiz:
  \begin{itemize}
  \item \url{http://software.uca.es/}
  \item \url{ftp://guadalinex.uca.es/Octave/}
  \item \url{http://softwarelibre.uca.es/Octave/}
  \end{itemize}
\end{itemize}

Por otro lado, \qtoctave cuenta con soporte técnico en nuestra
universidad, a través de la OSLUCA. Existe un foro dedicado
expresamente a ello: \url{http://osl.uca.es/forum}.

La mayoría de los usuarios de Windows y GNU/Linux preferirán utilizar
el programa ejecutable, ya compilado para su sistema, aunque los
usuarios más avanzados cuentan con la posibilidad de descargar y
compilar el código fuente.

\subsubsection*{Instalación en sistemas Windows}
En sistemas Windows, se recomienda utilizar el instalador disponible
(actualmente, \texttt{QtOctave-0.7.2.exe}). Éste guiará al usuario a
través de un sencillo proceso, al final del cual se instalarán tanto
Octave como \qtoctave, además de todos los paquetes adicionales o
\textit{toolkits} de octave-forge.

El instalador colocará un acceso directo en el escritorio a la vez que
una entrada de menú para \qtoctave. Utilizando cualquiera de ellos, se
puede arrancar el programa.


\subsubsection{Instalación en sistemas GNU/Linux}

Para instalar \qtoctave se necesita que estén instalados:

\begin{itemize}
\item Octave
\item Las bibliotecas de desarrollo de Qt(última versión). Generalmente llamadas libqt(version)-dev.
\item El compilador de C++
\end{itemize}

ntes de linux, recomendamos utilizar {\bf
  \qtoctave binario}. La instalación de \qtoctave binario se presenta
mucho más fácil de instalar, ya que sólo tenemos que dar los
siguientes pasos: 

\begin{enumerate}
\item Descargarnos la última versión de \qtoctave binario en la página (por ejemplo): https://forja.rediris.es/frs/download.php/745/qtoctave-0.7.4\_bin.tar.gz
\item Una vez descargado, tendremos que descomprimirlo y se nos aparecerá una carpeta con el nombre ``qtoctave-version\_bin''.
\item Dentro de esta carpeta, encontraremos otra carpeta y varios ficheros, entre ellos encontraremos un ejecutable llamado ``qtoctave.sh'', el cual tenemos dos formas de ejecutarlo:
\begin{itemize}
\item Dando doble click sobre el mismo
\item Abriendo una consola de comandos y tras situarnos mediante la orden ``cd'' en la carpeta donde se ubica, escribir la orden ``./qtoctave.sh''.
\end{itemize}
\end{enumerate}

\subsection*{El entorno \qtoctave}

\begin{figure}
\label{fig:qtoctave-secciones}
  \centering
  \pgfimage[width=0.8\linewidth]{figuras/qtoctave-secciones}
  \caption{Secciones de \qtoctave}
  
\end{figure}
La primera vez que iniciamos \qtoctave, tendremos un entorno
confortable,
(figura~\ref{fig:qtoctave-secciones} ),
en el que podemos distinguir distintas secciones:
\begin{enumerate}
\item \emph{Barra de menú}: Nos permite utilizar a algunas funciones
  de \octave mediante el ratón, utilizando ventanas de diálogo
  específicas para la introducción de parámetros.
\item \emph{Barra de iconos}: Facilita el acceso a algunas entradas
  del menú y las distintas a ventanas. Podemos agruparlos en cuatro
  bloques:
  \begin{itemize}
  \item Ejecución y control de procesos en la terminal.
  \item Acceso a la tabla para la introducción de matrices.
  \item Acceso a la ayuda
  \item Control de ventanas embebidas (ventanas de utilidades
    que están contenidas dentro de \qtoctave).
  \end{itemize}
\item \emph{Utilidades}: Ventanas de listado de variables, de lista de
  comandos y navegador de archivos.
  \begin{itemize}
  \item Lista de variables activas
  \item Lista de órdenes anteriores
  \item Navegador de archivos
  \end{itemize}
\item \emph{Terminal Octave}, o \emph{Consola}. Ventana para teclear
  comandos directamente en Octave.
\item \emph{Editor}: Para escribir programas que serán ejecutados por Octave.
\end{enumerate}

Habitualmente utilizaremos el editor y la consola para comunicarnos
con Octave, utilizando el lenguaje que se describe en las siguientes secciones.

%%% Local Variables: 
%%% mode: latex
%%% TeX-master: "smb"
%%% End: 
