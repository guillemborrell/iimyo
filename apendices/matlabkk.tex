\chapter{Lo que no me gusta de Matlab.}

Matlab no es ni mucho menos mi lenguaje de programación favorito.
Empecé con Matlab tras unos años de Fortran; poco tiempo después me
lancé con Python. Cuando estaba completamente emocionado con Python
me vi obligado a programar más en Matlab lo que me permitió compararlos
profundamente. Matlab salió perdiendo por goleada.\\

Además, cuando hago algún pequeño script uso el intérprete GNU/Octave.
Llevo muchos y felices años programando en entorno Linux donde la
instalación de Octave es trivial. Éste interprete no te deja hacer
muchas maravillas, simplemente soporta la parte matemática; nada de
gráficos renderizados y sombreados o interfaces de usuario. Para mí
es más que suficiente, en programas más complicados no lo uso.\\

Matlab es una gran ayuda para un ingeniero y es perfecto para cálculos
de complejidad media. Esto no impide que seamos críticos, Matlab tiene
defectos, algunos de ellos muy grandes; aunque donde yo digo defectos
alguien puede decir virtudes.


\section{Un lenguaje pobre y obsoleto.}

Matlab va sobre {}``trucos''. Fue muy frustrante ver como me era
imposible plasmar lo aprendido durante años con otros lenguajes. En
vez de crear estructuras cada vez más sofisticadas iba aprendiendo
cómo hackear el código descaradamente. El lenguaje no daba para más.
No es un lenguaje de programación general y no pretende serlo pero
tiene carencias que a mi parecer justificarían una revisión exhaustiva.
Lo peor de todo es que Mathworks va extendiendo el lenguaje a base
de añadidos que no tienen ninguna lógica. Por ejemplo la orientación
a objetos. ¿De qué sirve crear una clase si no hay absolutamente ninguna
en ningún toolkit? ¿Están diciendo desde Mathworks que quieren que
hagamos su trabajo?\\

Otro aspecto que me sorprendió muy negativamente fue la manera en
la que se manejaba la precisión numérica, sobretodo cuando se trabaja
con \texttt{Inf} o \texttt{0}. Matlab es incapaz de manejar integrales
impropias del modo correcto como lo consigue por ejemplo Python o
Java. La precisión de los cálculos internos es algo preocupante, además
de la velocidad. Cuando Matlab era el único lenguaje de scripting
científico decente aguantar su lentitud era algo asumido. Hoy en día
otros lenguajes de scripting son del orden de tres veces más rápidos.\\

El funcionamiento de Matlab es fuertemente imperativo y no soporta
ningún otro paradigma de programación; ni siquiera la programación
modular.  Todo es una función, incluso los métodos para pasar
variables entre unidades de programa. Este paradigma resulta
especialmente molesto cuando los programas se complican. Mantener la
sintaxis del lenguaje minima e ir añadiendo una función tras otra no
es siempre la solución porque tiende a empobrecer el lenguaje y obliga
a aprender centenares
de funciones que uno utiliza una vez en su vida.\\

El hecho de que por cada función definida del modo usual se tenga
que crear un archivo me saca de quicio. Si eso desembocara en una
ganancia espectacular de velocidad estaría hasta cierto punto justificado,
pero no es así. Es una tarea cansina partir un algoritmo en decenas
de archivos porque ir juntando las piezas en function handles convierten
el código en un jeroglífico. También lo es tener que jugar con las
variables para pasar argumentos a funciones de una manera eficiente
o crear estructuras de variables artificiales porque a Matlab le faltan
tipos como listas o diccionarios. No se pueden usar estructuras complejas
de datos porque no se pueden hacer asignaciones por grupo, no se pueden
crear iteradores ni iterar sobre listas y muchas cosas más. En una
comparación con otros lenguajes de scripting como Python o Ruby, Matlab
sale netamente perdedor.


\section{DWIMBNWIW.}

Estas son las iniciales de Do What I Mean But Not What I Write. Es
un paradigma de los lenguajes de programación en el que el resultado
de nuestro código no depende del modo en el que se ha escrito. Significa
que el compilador o el intérprete es tan listo que entiende lo que
queremos decir aunque no hayamos sido capaces de plasmarlo en un código
óptimo. Podemos preocuparnos más de escribir un código elegante y
leíble sin pensar demasiado en una optimización máxima. Fortran es
quizás el lenguaje de programación que se acerca más a este objetivo.\\

Matlab hace literalmente lo que escribimos sin ningún tipo de optimización
automática. La consecuencia inmediata es que, tal como hemos visto
durante los ejemplos en el libro, la velocidad de ejecución depende
directamente de nuestro estilo de programación. Un bucle mal puesto
o una función sin optimizar puede aumentar el tiempo de ejecución
en varios ordenes de magnitud. La optimización {}``en directo''
no es algo ni imposible ni nuevo en lenguajes interpretados. Optimizar
los bucles no es tan difícil y en Matlab sólo se hace si compilamos
las funciones.\\

La consecuencia indirecta es que la escritura del programa queda condicionada
por el resultado. \textbf{Este defecto es un motivo suficiente como
para calificar a Matlab de mal lenguaje de programación}. Algunos
lenguajes imponen un estilo pero nunca por motivos relacionados con
el resultado y siempre por elegancia o coherencia. Para que un código
en Matlab funcione bien debemos evitar los bucles... ¿Cómo pueden
pedirnos que evitemos una de las dos estructuras esenciales de la
programación? Para conseguir el máximo rendimiento debemos encapsular
los comandos lo máximo posible... Al final el código es difícilmente
leíble. Para evitar las pérdidas de memoria debemos utilizar la menor
cantidad de variables posibles... ¿No sería mejor dotar a Matlab de
automatic garbage collection%
\footnote{En un lenguaje donde se van creando y destruyendo variables y donde
las funciones reciben los argumentos por valor en vez de por referencia
es muy útil tener un mecanismo que libere la memoria que no se vuelve
a utilizar. Automatic Garbage Collection es el nombre que recibe el
mecanismo de búsqueda y eliminación de variables intermedias inútiles.
Matlab sólo dispone del comando \texttt{clear}.%
}?


\section{Matlab para todo, absolutamente todo.}

Cuando aprendí programar de verdad fue cuando dominé más de un lenguaje
de programación. Fortran me maravilla porque en ningún momento ha
querido salir del ámbito científico; si quieres herramientas gráficas
programalas en C++ o en Java. En cambio Matlab te permite hacer absolutamente
de todo, pero mal. Un ejemplo es el módulo para la creación de interfaces
gráficas. Un programador experimentado sabe que cuando programa un
interfaz lo más importante es saber en qué toolkit lo está haciendo
porque puede que no funcione bien en algunos ordenadores. Un ejemplo
claro son las máquinas Linux o BSD donde no hay una máquina virtual
de Java por defecto. ¿Cuál es el toolkit de Matlab? Para programar
interfaces gráficas es muy conveniente irse a otro lenguaje. De acuerdo,
se puede hacer en Matlab; pero no convencerán a nadie si dicen que
es la solución más adecuada. Hoy en día no se concibe una API para
interfaces gráficas que no soporte orientación a objetos. Otro ejemplo
es el toolkit de proceso de imágenes, muy inferior a Imlib que tiene
bindings a casi todos los lenguajes de programación (excepto Matlab),
es gratuita, rápida y con una API deliciosamente bien escrita.\\

El ingeniero {}``Matlab-monkey'' es un patrón que se está extendiendo
peligrosamente. Matlab puede hacer de todo, bien o mal; esto acaba
muchas veces con el sentido crítico de que siempre hay que escoger
la herramienta adecuada para una tarea. Para procesar datos o para
pequeños scripts matemáticos Matlab es simplemente perfecto. Para
trabajar con tipos complejos, crear grandes estructuras lógicas o
hacer cálculos agresivos con la precisión del ordenador debemos buscar
otra solución.\\

El fenómeno {}``Matlab-monkey'' nace de la facilidad con la que
se programa en Matlab y es agravado por la creencia que aprender un
lenguaje de programación nuevo es costoso e inútil. Todos los programadores
que dominen, y con dominar me refiero a ser capaces de escribir programas
de miles de lineas que funcionen, admiten que usar regularmente tres
o cuatro lenguajes de programación dota de unos recursos impagables;
en el fondo todos son iguales.


\section{El poder del dinero}

En el tema de licencias Microsoft es una hermanita de la caridad al
lado de Mathworks. Matlab es un programa caro, muy caro. Si compramos
alguna versión de estudiante a un precio económico nos podemos encontrar
con la desagradable sorpresa de que no está la función \texttt{fsolve}.
El motivo es que Matlab está dividido en multitud de toolkits por
motivos económicos y algunos de los esenciales están ausentes en el
paquete básico. Ahora pensamos que no hay ningún problema, que siempre
se puede conseguir pirateado... Pero si tenemos una empresa no hay
pirateo que valga, hay que comprar el software. Un Matlab completamente
operativo puede costarnos:

\begin{description}
\item [\texttt{2400\EUR}]Matlab. Sólo el intérprete, las funciones
  básicas y la interfaz gráfica.
\item [\texttt{1150\EUR}]Optimization toolkit. Sin esto no tenemos
  \texttt{fsolve} .
\item [\texttt{1000\EUR}]Statistics toolkit. Sin esto no tenemos
  funciones estadísticas
\item [\texttt{1000\EUR}]Signal processing toolkit. Sin esto nada de
  convoluciones y FFT.
\end{description}
Y si encima queremos el Matlab Compiler para que el código vaya rápido
de verdad (\texttt{6250}\EUR), Simulink (\texttt{3500}\EUR) y
Stateflow (\texttt{3500}\EUR) la suma puede salirnos por la friolera
de \texttt{18800}\EUR.  Y más nos vale ponerlo en un buen servidor de
aplicaciones porque esto es el coste por puesto de trabajo.\\

Si lo comparamos con el coste cero de muchas de los programas que
podemos encontrar por Internet, algunas de ellos de tanta calidad
o más que Matlab, es probable que cambiemos de herramienta de uso
cotidiano.


\section{Python}

Programo en Matlab a menudo pero esto no significa que sea mi lenguaje
favorito. Aprecio su carácter directo y su sencillez pero cuando tengo
que programar aplicaciones más serias me voy a otro lenguaje interpretado,
Python. Tiene absolutamente todo lo que se le puede pedir a un lenguaje
de programación, además está creciendo en adeptos. Es muy recomendable
echarle un vistazo, probablemente alguien vea la luz con Python; incluso
puede ser que descubra que programar es divertido.\\

El módulo orientado a matemáticas más versátil y completo es SciPy.
Aunque está aún incompleto se ha convertido en una referencia entre
los programadores en Python. A los recién llegados a este lenguaje
les resulta raro encontrar rutinas para funciones de interpolación
implementadas con orientación a objetos. Una vez se supera la sorpresa
inicial el código resultante se vuelve cada vez mejor, incluso puede
volverse más directo que Matlab.
