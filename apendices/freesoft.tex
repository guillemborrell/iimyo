\chapter{Software y formatos libres.}

Para escribir este curso he usado una de las herramientas de
publicación más potentes que hay en la actualidad, \TeX{}. Es un poco
más conocida una colección de macros llamada \LaTeX{}. No he escrito
directamente código \LaTeX{} sino que he usado un editor de textos
llamado LyX.  La calidad del documento final es apreciable; las
fuentes son claras y la tipografía es perfecta. Probablemente a alguno
le sorprenda saber que estas herramientas son gratuitas. No me refiero
a que las haya pirateado ni que me las hayan regalado; podemos
descargarnos todos estos programas e instalarlos en nuestro ordenador
de forma gratuita (siempre que respetemos la licencia adjunta). Además
no solo podemos descargar el programa, sino que el código fuente
también está disponible
con una licencia que permite su modificación y su redistribución.\\


\TeX{} es quizás el ejemplo más llamativo por ser el primero. Lo
programó Donald E. Knuth porque estaba muy descontento con la calidad
de la tipografía de algunas editoriales con las que tenía tratos. A
partir de entonces la cantidad de proyectos cuya máxima es la
posibilidad de compartir y redistribuir el código libremente no ha
hecho más que aumentar.  Otros ejemplos conocidos (algunos más otros
menos) son Linux, implementación libre y gratuita de un núcleo de
sistema operativo tipo UNIX; Emacs, posiblemente el editor más
completo y extensible que existe; Apache, el servidor web más usado;
PostgreSQL, una base de datos muy popular; KDE, un entorno de
escritorio completo que rivaliza en calidad con WindowsXP
y Mac OSX...\\


El software libre tiene dos grandes ventajas respecto al comercial, el
coste cero sin garantía y (ventaja poco valorada) ser proyectos
llevados en comunidad. Una vez un programa comercial sale al mercado
se independiza completamente de sus desarrolladores; el mantenimiento
pasa a ser labor del servicio técnico. Si el programa funciona mal o
de un modo distinto al supuesto habrá que llamar por teléfono y
solicitar ayuda. Una comunidad es una estructura completamente
distinta.  Supongamos que encontramos un bug en Octave, uno grave que
no nos permita seguir con nuestro trabajo. ¿A quién llamamos? No hay
ninguna empresa que ofrezca garantía, además en la licencia pone bien
claro que Octave no es un producto mercantil y por tanto no está
sujeto a ninguna normativa concreta. La frase usada es \emph{use this
  software at your own risk}. Lo que haremos será ir a google e
investigar si alguien ha tenido antes el mismo problema; si la
respuesta es negativa pasaremos a las listas de correo. Las listas de
correo son la base de una comunidad de desarrolladores junto con los
canales IRC. Para quien no lo sepa una lista de correo es una
dirección de correo electrónico que manda lo que recibe a todos los
miembros de la lista. Si vemos que somos los primeros en detectar el
fallo debemos suscribirnos a la lista de usuarios de Octave (bastante
activa, por cierto), describir el fallo y pedir amablemente que
alguien nos ayude. Si tenemos suerte es probable que John W. Eaton,
creador de Octave, nos responda. Si además le damos nosotros la
solución y esto sucede unas cuantas veces puede ser que nos de permiso
de escritura en el repositorio de código y que nos de una cuenta de
desarrollador. En Octave todo se hace a pequeña escala; en proyectos
grandes como KDE el proceso de desarrollo
involucra a unas mil personas y varias empresas.\\


Este fenómeno ha crecido gracias a dos pequeños detalles, el uso de
licencias \emph{copyleft} y el uso de estándares abiertos. He
publicado este texto bajo una licencia Creative Commons, creada por
una fundación cuyo fin es la publicación de licencias de distribución
\emph{copyleft}.  Lo he hecho por la creencia personal de que el
conocimiento debe ser público y gratuito. Este afán altruista puede
provocar consecuencias bastante desagradables en el caso que no sea
perfectamente planeado.  Es aquí donde entran las licencias
\emph{libres} o de \emph{copyleft}.  Una licencia \emph{libre} asegura
la máxima libertad de uso; es un blindaje comercial absoluto que puede
ser tan bebeficioso en algunos casos como perjudicial en otros. Luego
nos encontramos varias licencias que limitan las libertades de uso,
siempre intentando preservar primero los derechos del autor. En este
caso he utilizado una licencia no comercial pero sin posibilidad de un
trabajo derivado. Esto significa que cedo la libertad de copiar
literalmente el texto sin ningún coste económico pero que me reservo
el derecho de cobrar por ello cuando quiera. También impide que
cualquier otro autor que copie partes del
libro pueda hacerlo sin citarme.\\


Cuando alguien con este afán altruista, normalmente acompañado de un
desprecio total por el lucro, pretende programar una aplicación o
publicar un libro en comunidad nunca inventará su propio formato de
distribución. Lo que hará será buscar el formato más extendido para
que su trabajo llegue a la mayor cantidad de gente posible. Esto
está asegurado con el uso de estándares abiertos.\\


Los estándares abiertos son tipos de archivos o protocolos de
comunicación cuyas especificaciones técnicas se han publicado enteras.
Ejemplos de estándares abiertos son los protocolos TCP/IP, HTTP, FTP,
SSH; los lenguajes C, C++, Python; los tipos de documentos HTML, SGML,
RTF, PS, PDF, OD{*}... Permiten que dos programas que no tienen
ninguna línea de código en común puedan entenderse a la perfección. El
software libre o de código abierto (hay una diferencia muy sutil entre
los dos) utiliza casi siempre estándares abiertos porque es absurdo no
hacerlo. Cuando alguien publica un programa con su propio estándar
debe hacerlo desde una posición de fuerza para conseguir imponerlo.\\


He aquí el gran enemigo del software libre, los estándares cerrados.
Los ejemplos de estándares cerrados más manifiestos son los impuestos
por Microsoft. Cada producto que sacan al mercado tiene su propio
formato de archivo incompatible con los ya existentes. Es una práctica
monopolística muy habitual y las consecuencias han sido nefastas para
el mundo de la informática. Sólo hace poco han decidido publicar parte
de las especificaciones porque así lo han requerido varios gobiernos
por motivos de seguridad nacional.\\


\textbf{Todos los formatos cerrados son nocivos; cuanto más extendidos
  más perjudiciales. El archivo .doc ha hecho un daño irreparable.
  Microsoft Word es el único procesador de textos existente que no
  exporta a PDF.  ¿Por qué? Porque Microsoft intenta imponer el
  archivo DOC para distribución de archivos de texto. Si uno usa
  Windows en su ordenador es su problema, si usa Explorer allá él pero
  si usa Microsoft Office el problema es de todos.} Utilizar un
programa de software libre es una garantía de uso de estándares
abiertos. Si uno prefiere seguir utilizando productos de Microsoft que
consiga un plugin de salida a PDF o que use el formato RTF (Rich Text
Format). Todos los que no seamos esclavos de Microsoft
se lo vamos a agradecer eternamente.\\


No estoy sugiriendo que todo el mundo instale Linux en su ordenador
(aunque así viviríamos en un mundo mejor). Lo que sí recomiendo
encarecidamente es que se usen formatos abiertos para compartir
documentos. Hace muy poco se publicaron las espeficicaciones técnicas
de la alternativa a los formatos de Microsoft Office llamados Open
Document. Los programas que lo soportan son, de momento, OpenOffice y
Koffice. OpenOffice es una suite de ofimática muy completa de un invel
equivalente a Microsoft Office. Tiene versiones nativas para Linux,
Windows, Solaris y Mac OSX. Koffice es la suite particular de KDE y es
un proyecto a una
escala menor.\\


Otro ejemplo flagrante es el del estándar html. Es el lenguaje de
marcas en el que están escritas las páginas web. Todos los navegadores
están diseñados para soportar todas las especificaciones del estándar
excepto uno. Esta oveja negra es el Microsoft Internet Explorer.
Además los desarrolladores de este navegador han asegurado que no
pretenden soportar enteramente el formato html más actual en un futuro
cercano.  Esto obliga a los desarrolladores de páginas web a utilizar
las especificaciones del html soportado para el Internet Explorer y no
las que marca el estándar de la w3c, organización sin ánimo de lucro
que vela por los estándares de publicación en Internet. Esta situación
vergonzosa es provocada por la manifiesta falta de calidad del
navegador de Microsoft y por la aplicación de técnicas monopolísticas
en el mercado de el diseño de páginas web. ¿Cuántas páginas en el
mundo están optimizadas
para el Internet Explorer? ¿Cuántas llevan el sello de w3c?\\


Espero que este apéndice ayude a concienciar a alguien de lo
importante que es no ceder al chantaje de ciertas empresas cuyo único
objetivo es enriquecerse a costa manipular al consumidor.
