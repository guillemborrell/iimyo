


%%%%%%%%%%%%%%%%%%%%%%%%%%%%%%%%%%%%%%%%%%%%%%%%%%
\chapter{Guía de estilo. Los 10 mandamientos}

%%%%%%%%%%%%%%%%%%%%%%%%%%%%%%%%%%%%%%%%%%%%%%%%%%

En muy bueno para un lenguaje de programación que tenga una guía de
estilo asociada. Cómo formatear el código debe ser algo más que una
marca de estilo personal, un estándar que todo el mundo siga a
rajatabla.

En algunos lenguajes la sintaxis viene acoplada a unas ciertas reglas
de escritura, por ejemplo Python nos obliga a tabular las estructuras;
si no ordenamos bien el inicio de las líneas el código no funcionará.
C, C++ y Java usan corchetes para agrupar niveles de ejecución y se
recomienda dejar sólo el corchete en la línea y tabular todo lo que
contenga.

En muchos casos estas reglas son en realidad recomendaciones. En
Fortran recomiendan marcar los niveles de ejecución con tabulados de
dos espacios, algo parecido se recomienda en Matlab. Pero tanto en
Fortran como en Matlab estas normas son sistemáticamente ignoradas. El
programador \emph{científico} no ha recibido ninguna educación sobre
las reglas de escritura y tiene muy poca experiencia con lenguajes más
fuertemente estructurados como C o Java. El resultado es un código
plano, desestructurado, desorganizado y mal comentado.

Una de las herramientas esenciales para evitar este desastre es el
editor de texto. Sería muy largo describir el funcionamiento de un
editor de texto, simplemente diremos que es un programa que edita
código fuente y que tiene todas las herramientas disponibles para que
sea mucho más fácil. Colorea las palabras clave, tabula el código
automáticamente, corrije algunos fallos comunes... Matlab posee un
buen editor de textos, pero no le llega ni a la suela del zapato a
Emacs o VIM. El funcionamiento de Emacs puede resultar hasta
sorprendente.  Imaginemos que abrimos un archivo de código escrito en
Fortran 95.  Está lleno de \texttt{if} y \texttt{do} sin ordenar y mal
tabulado.  Además el compilador nos dice que hay una estructura que
está sin cerrar; que nos falta un \texttt{end}. Si estamos editando el
archivo con Emacs sólo tenemos que pulsar \texttt{<Alt><Control><Q>} y
todo el texto se tabula automáticamente.

Sin embargo el estilo de un buen programador es insustituible. Matlab
no tiene ninguna guía de estilo, sin embargo puede decirse que existen
una serie de normas no escritas para que el código sea más leíble y,
no menos importante, ejecute rápido.

Estas normas parecen inservibles para pequeños scripts que escribimos
para consumo propio. He aprendido que cuando se recupera un script que
se ha escrito seis meses antes es como si lo hubiera escrito otro.
Siempre que escribamos un programa tomaremos esta actitud; lo estamos
escribiendo para que un desconocido lo entienda.

Todas las buenas prácticas se resumen en diez técnicas, los 10
mandamientos del Matlab.

\begin{description}
\item [1.]Utilizar funciones \texttt{inline} y \texttt{@} siempre que
  se pueda.
\end{description}
Ya hemos visto que en Matlab (que no en Octave) definir una nueva
función significa crear un archivo. Esto hace que el código de el
programa principal sea más difícil de entender y de comentar. Aunque
Matlab escala bien no es muy recomendable escribir programas demasiado
largos y si tenemos demasiadas funciones y archivos es que no estamos
haciendo las cosas bien.

Las sentencias \texttt{inline} y los \emph{function handles} nos
ayudan a mantener el código compacto y leíble.

\begin{description}
\item [2.]Usar bucles \texttt{for} sólo donde sea indispensable
\end{description}
Esta es la gran diferencia entre Matlab y Fortran. Una de las grandes
decepciones de los programadores en Fortran cuando utilizan Matlab es
la gran diferencia de velocidad. Esta diferencia existe y es de entre
uno y dos órdenes de magnitud. El problema es que es gravemente
acentuada por usar el estilo de Fortran en el código Matlab. En
Fortran los bucles \texttt{do} son endiabladamente rápidos. Cuando se
compila el código los contadores se optimizan como estructura entera,
incluso algunos compiladores advierten cuando los bucles encapsulados
no son dependientes y paralelizan la tarea automáticamente. El
programador en Fortran encapsula porque sabe que el compilador le va a
ayudar.

Matlab no es capaz de optimizar el código de esta manera porque su
arquitectura no lo permite. La solución para obtener una velocidad
aceptable es usar siempre submatrices.

\begin{description}
\item [3.]Usar estructuras \texttt{if} planas.
\end{description}
Cuando hablamos de condicionales dijimos que el encapsulado tipo
\emph{matroska} no era recomendable porque complica mucho la lectura
del código. Además disminuye la velocidad de ejecución por la creación
de más variables adicionales. Si utilizamos condiciones lógicas
compuestas (siempre dentro de un límite) el resultado será más leíble
y posiblemente más rápido.

\begin{description}
\item [4.]Utilizar las funciones de creación de matrices.
\end{description}
Las funciones como \texttt{zeros}, \texttt{ones}, \texttt{eye},
\texttt{linspace} o \texttt{meshgrid} son tremendamente útiles. El
código de un buen programador estará siempre lleno de composición de
estas funciones para generar matrices. La alternativa es utilizar
bucles, prohibido por el segundo mandamiento.

\begin{description}
\item [5.]Definir las variables de las cabeceras \emph{in situ}.
\end{description}
Cuando se pasan los argumentos a una función lo habitual es crear
primero las variables que los contienen y luego poner las variables en
la cabecera. Si lo hacemos es casi obligado hacerlo justo antes de
llamar a la función. De otro modo tendremos que navegar por el código
hasta encontrar dónde se define la variable que pasa el argumento.

Es recomendable también que las constantes o los argumentos simples se
definan directamente en la cabecera. Puede parecer que el resultado es
difícil de leer, pero cuando uno se acostumbra resulta ser todo lo
contrario. Es mucho mejor una línea con varios comandos que no
repartir sentencias asociadas por todo el código.

\begin{description}
\item [6.]Utilizar submatrices siempre que se pueda.
\end{description}
Este es el \emph{No matarás} de los mandamientos de Matlab. La
alternativa es utilizar bucles con un puntero que vaya rastreando la
matriz; demasiado lento. El equivalente de esta recomendación en
Fortran sería utilizar el \texttt{forall} en vez de los bucles
simples.

Esta práctica sólo tiene ventajas. El código es más leíble, es plano,
mejor encapsulado, se entiende perfectamente y es más rápido. Cuando
se dice que Matlab es un programa de cálculo matricial no se dice
gratuitamente. No hay que pensar contando índices con la notación del
sumatorio como en Fortran, hay que pensar matricialmente.

\begin{description}
\item [7.]Tabular bien los bloques de ejecución
\end{description}
Aunque ya hemos aprendido que lo más importante es evitar el
encapsulado de estructuras de ejecución en algunos casos no tendremos
más salida.  En ese caso el tabulado de cada uno de los bloques es muy
recomendable.  Además de obteber un resultado más leíble evitaremos
los errores tontos como los bucles o los condicionales sin cerrar.

\begin{description}
\item [8.]Comentar cualquier operación que lo precise.
\end{description}
Lo repito, cuando recuperemos código escrito unos meses antes será
como si lo hubiera escrito otro. puede ser que una sentencia
especialmente brillante sea además críptica; comentaremos
sistemáticamente todo lo que sea necesario.

\begin{description}
\item [9.]Escribir una cabecera de ayuda en cada función.
\end{description}
Este mandamiento es directamente derivado del octavo. A medida que
escribamos funciones terminaremos con una pequeña biblioteca personal.
El resultado será que tendremos que usar funciones sin recordar ni la
forma de su cabecera. El comando \texttt{help} es imprescindible no
sólo para las funciones de la biblioteca; también nos servirá en las
que escribamos nosotros mismos.

\begin{description}
\item [10.]Usar Octave en vez de Matlab y colaborar en el proyecto.
\end{description}
Las soluciones más justas no siempre son las más fáciles. Octave es
más espartano y difícil pero ha sido posible gracias al esfuerzo de
muchos desarrolladores que han colaborado desinteresadamente en el
proyecto. El resultado es un programa de gran calidad que no para de
mejorar día a día.

El hecho de que se trate de software libre es un valor añadido.
Podremos escudriñar en sus entrañas para descubrir su funcionamiento,
extenderlo con otros lenguajes de programación, ayudar en proyectos
secundarios.  Una vez se entra en este juego no se para de aprender.


\subsubsection*{El último mandamiento que los recoje a todos:}

\begin{center}

Escribirás lo menos posible usando todas las funciones que puedas.\end{center}
